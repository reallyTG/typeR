\documentclass[acmsmall,review,anonymous]{acmart}\settopmatter{printfolios=true,printccs=false,printacmref=false}
\acmJournal{PACMPL}
\acmVolume{1}
\acmNumber{OOPSLA} % CONF = POPL or ICFP or OOPSLA
\acmArticle{1}
\acmYear{2018}
\acmMonth{1}
\acmDOI{} % \acmDOI{10.1145/nnnnnnn.nnnnnnn}
\startPage{1}
\setcopyright{none}
\bibliographystyle{ACM-Reference-Format}
\citestyle{acmauthoryear}   %% For author/year citations
\usepackage{my_style}
\usepackage{listings, wrapfig,xspace}

\lstset{language=R}
\definecolor{LightGray}{rgb}{.92,.92,.92}
\definecolor{Gray}{rgb}{.3,.3,.3}
\definecolor{DarkGray}{rgb}{.5,.5,.5}
\lstset{ %
  columns=flexible,
  captionpos=b,
  frame=single,
  framerule=0pt,
  tabsize=2,
  belowskip=0.5em,
  backgroundcolor=\color{LightGray},
  basicstyle=\small\ttfamily,
  emphstyle=,
  keywordstyle=,
  commentstyle=\color{Gray}\em,
  stringstyle=\color{Gray}
}
\lstdefinestyle{R}{ %
  language=R,
  morekeywords={assign, delayedAssign},
  deletekeywords={env, equal, c, runif, trace, args, exp, t, all},
  breaklines=true
}
\lstdefinestyle{Rin}{ %
  style=R,
  breaklines=false
}

\newcommand{\eg}{\emph{e.g.},\xspace}
\newcommand{\ie}{\emph{i.e.},\xspace}

\newcommand{\PIR}{\textsf{PIR}\xspace}
\newcommand\pirI[1]{\mathtt{#1}}
\renewcommand{\c}[1]{{\lstinline[style=Rin]!#1!}\xspace}
\newcommand{\code}[1]{{\lstinline[style=Rin]!#1!}\xspace}
% Macros for type names from the old paper.
\newcommand{\attr}[2]{\ensuremath{#1_{\mathtt{#2}}}\xspace}
\newcommand{\attrclass}[3]{\ensuremath{#1^{\mathtt{#3}}_{\mathtt{#2}}}\xspace}
\renewcommand{\to}{\ensuremath{\rightarrow}\xspace}
\newcommand{\D}{\ensuremath{\small\vec{\mathtt D}}\xspace} % Double
\newcommand{\I}{\ensuremath{\small\vec{\mathtt I}}\xspace} % Integer
\renewcommand{\C}{\ensuremath{\small\vec{\mathtt C}}\xspace} % Character
\renewcommand{\L}{\ensuremath{\small\vec{\mathtt L}}\xspace} % Logical
\newcommand{\R}{\ensuremath{\small\vec{\mathtt R}}\xspace} % Raw
\newcommand{\X}{\ensuremath{\small\vec{\mathtt X}}\xspace} % Complex
\newcommand{\Y}{\ensuremath{\small\vec{\mathtt Y}}\xspace} % Symbol
\newcommand{\sY}{\ensuremath{\small{\mathtt Y}}\xspace} % Symbol
\newcommand{\sS}{\ensuremath{\small{\mathtt S}}\xspace} % S4
\newcommand{\sF}{\ensuremath{\small{\mathtt F}}\xspace} % Closure
\newcommand{\sE}{\ensuremath{\small{\mathtt E}}\xspace} % Env
\renewcommand{\R}{\ensuremath{\small\vec{\mathtt R}}\xspace} % Raw
\newcommand{\sN}{\ensuremath{\small{\mathtt N}}\xspace}     % Null
%\renewcommand{\l}{\ensuremath{\small L<?>}\xspace}     % List
\renewcommand{\l}{\ensuremath{\small\underline{\mathtt ?}}\xspace}     % List
\newcommand{\sD}{\ensuremath{\small{\mathtt D}}\xspace} % Double
\newcommand{\sI}{\ensuremath{\small{\mathtt I}}\xspace} % Integer
\newcommand{\sC}{\ensuremath{\small{\mathtt C}}\xspace} % Character
\newcommand{\sL}{\ensuremath{\small{\mathtt L}}\xspace} % Logical
\newcommand{\sX}{\ensuremath{\small{\mathtt X}}\xspace} % Complex
\newcommand{\sR}{\ensuremath{\small{\mathtt R}}\xspace} % Raw
\newcommand{\ANY}{\ensuremath{\small{\mathtt ?}}\xspace}     % Any
\newcommand{\lT}[1]{\ensuremath{\small\underline{\mathtt{#1}}}\xspace}     % list<T>
\newcommand{\M}[1]{\ensuremath{\attr{\vec{\tt #1}}{mat}}\xspace}     % matrix
\newcommand{\df}{\ensuremath{\attr{\l}{df}}\xspace}     % data.frame

\newcommand{\contractr}{\emph{contractr}\xspace} % contractr
\newcommand{\roxygen}{\emph{roxygen2}\xspace} % roxygen
\newcommand{\typetracer}{\emph{typetracer}\xspace} % typetracer

\begin{document}
\title{Types for R, Empirically Designed and Evaluated}
\begin{abstract}

\end{abstract}
\maketitle


\section{Introduction}

The story that we're aiming for:
Designing type systems for established languages can be a challenging.
We aim to approach this problem by essentially reverse-engineering a type system from a corpora analysis of a representative subset of R code.
We present a simple type system that captures the vast majority of usage of R --- it is lightweight, and \AT{should provide enough information to be useful to a compiler}.

To evaluate our type system, we present a type-checker that is aware of R's promise semantics. 
In R, function arguments are wrapped in promises such that they are not evaluated until they are needed.
Our tool deals with these promises by inserting a check which fires only when the promise is forced, at which point the value is checked for compliance with the type.

\AT{I can grab some of the intro Jan wrote for the last paper, rework it, and put it here.}

%
%
%
%
%
%
\section{Background}

In this section, we will introduce the R programming language, as well as discuss the dynamic analysis framework we used as part of our initial analysis.

%
%
%
%
\subsection{The R Programming Language}
\label{sec:R}

Over the last decade, the R Project has become a key tool for implementing
sophisticated data analysis algorithms in fields ranging from Computational
Biology~\cite{R05} to Political Science~\cite{R:Keele:2008}. At the heart of
the R project is a \emph{vectorized, dynamic, lazy, functional,
  object-oriented} programming language with a rather unusual combination of
features~\cite{ecoop12} designed to ease learning by non-programmer and
enable rapid development of new statistical methods.  The language, commonly
referred to as R was designed in 1993 by Ross Ihaka and Robert
Gentleman~\cite{R96} as a successor to S~\cite{S88}.  First released in
1995, under a GNU license, R rapidly became the lingua franca for
statistical data analysis. Today, there are over 13,000 R packages available
from repositories such as CRAN and Bioconductor.  With 55 R user groups
world-wide, Smith~\cite{eco11} estimates that there are over 2 million
end-users.

As an introduction to R, consider the code snippet in Fig.~\ref{sample} from
a top-level interaction where the user defines a function \code{normSum}
that accepts vectors of integers, logicals, doubles and complex values and
normalizes the vector with respect to its sum and rounds the results. The
function definition does not require type annotations, and all operations
transparently work on vectors of any length and different types.

\begin{figure}[!hb]{\small
\begin{lstlisting}
> normSum <- function( m )  round( m / sum(m), 2)
> normSum(c(1L,3L,6L))
[1] 0.1 0.3 0.6
> normSum(c(1.1,3.3,6.6))
[1] 0.1 0.3 0.6
> normSum(c(1.6,3.3,6.1))
[1] 0.15 0.30 0.55
> normSum(complex(r=rnorm(3),i=rnorm(3)))
[1] 0.49+0.21i 0.30-0.18i 0.22-0.03i
\end{lstlisting}}
\caption{Sample R code}\label{sample}
\end{figure}

In R, function can be called with named parameters, R support variable
argument lists, and arguments can have default values. Putting all of these
together consider the following declaration:

\begin{lstlisting}
f <- function(x, ..., y=3) x + y
\end{lstlisting}

\noindent
Function \k{f} can be called with a single argument \code{f(3)}, with named
argument \code{f(y=4,x=2)} and with a variable number of arguments,
\code{f(1,2,3,4,y=5)}, all of these calls will return \code{6}.

R has a number of features that are not crucial to the present
discussion. We will mention some of them here for completeness.  In R, data
structures are reference counted and have copy-on-write semantics, thus the
assignment \code{x[12]<-3} results in an update to a copy of \code{x} unless
the reference count on that object is 1.  This semantics gives R a
functional flavor while allowing updating in place within loops (the first
update copies, subsequent updates are performed on the copy). Arguments to
functions are evaluated only when needed, they are bundled in so-called
promises which package the original expression (as an abstract syntax tree, or AST), its environment
as well as the result of evaluating the expression. Promises can be
leveraged for meta-programming as it is possible to retrieve the text of a
promise and evaluate it in a different environment.


\subsubsection{Types of Data}
\label{subsubsec:backgroundtypes}

Before attempting to define a type system for R, we should understand the
different kinds of values that programs operate on.  As we will see
different notions of type may emerge depending on how granular we want to
be.

\renewcommand{\k}[1]{{\tt #1}\xspace}

R has one builtin notion of type that can be queried by the \k{typeof}
function. Over the years, programmers have found the need for a richer type
structure and have added {\it attributes}. The best way to think of attributes is
as an optional map from name to values that can be attached to any object.
Attributes are used to encode various type structures. They can be queried
with functions such as \k{attributes} and \k{class}.

\begin{wrapfigure}{r}{6.1cm}
\footnotesize\begin{tabular}{l|c|l@{}}\hline
\multicolumn{3}{l}{\bf Vectorized data types:}  \\\hline
\k{logical}   & \L & vector of boolean values\\
\k{integer}   & \I & vector of 32 bit integer values\\
\k{double}    & \D & vector of 64 bit floating points\\
\k{complex}   & \X & vector of complex values\\
\k{character} & \C & vector of strings values\\
\k{raw}       & \R & vector of bytes\\
\k{list}      & \l & vector of values of any type\\\hline
\multicolumn{3}{l}{\bf Scalar data types:}\\\hline
\k{NULL}      & \sN &  singleton null value\\
\k{S4}        & \sS &  instance of a S4 class \\
\k{closure}   & \sF & a function with its environment\\
\k{environment}&\sE &  a mapping from symbol to value \\\hline
\multicolumn{3}{l}{\bf Implementation data types:}\\\hline
\multicolumn{3}{l}{\k{special},
\k{builtin},
\k{symbol} (\sY),
\k{pairlist},
\k{promise}}\\
\multicolumn{3}{l}{
\k{language},
\k{char},
\k{...},
\k{any},
\k{expression},
}\\
\multicolumn{3}{l}{
\k{externalprt},
\k{bytecode},
\k{weakref}}\\\hline
\end{tabular}\caption{Builtin Types}\label{types}\end{wrapfigure}

Figure~\ref{types} lists all of the builtin types that are provided by the
language. They are the possible return values of \k{typeof}. There is no
intrinsic notion of subtyping in R. But, in many context a \k{logical} will
convert to \k{integer}, and an \k{integer} will convert to \k{double}.  Some
off conversion can occur in corner cases, such as \k{1<"2"} holds and
\k{c(1,2)[1.6]} returns the first element of the vector, as the double is
converted to an integer. R does not distinguish between scalars and vectors
(they are all vectors), so \code{typeof(5) ==} \code{typeof(c(5)) ==
  typeof(c(5,5))} \code{ == "double"}. Finally all vectorized data types have a
distinguished missing value denoted by \code{NA}. The default type of
\code{NA} is \k{logical}. We can see that \code{typeof(NA)=="logical"}, but
NA inhabits every type: \code{typeof(c(1,NA)[2])=="double"}.

With one exception all vectorized data types are monomorphic, the exception
is the \k{list} type which can hold values of any other type including
\k{list}. For all monomorphic data types, attempting to store a value of a
different type will cause a conversion. Either the value is converted to the
type of vector, or the vector is converted to the type of the value.

Scalar data types include the distinguished \k{NULL} value, which is also of
type \k{NULL}, instance of classes written using the S4 object system,
closures and environments.  The implementation of R has a number of other
types that are mostly not used by user code, they are listed in
Figure~\ref{types} for reference.

The addition of attributes lets programmers extend the set of types by
tagging data with user-defined attributes. For example, one could define a
vector of four values, \code{x<-c(1,2,3,4)} and then attach the attribute
\k{dim} with a pair of numbers as value: \code{attr(x,"dim")<-c(2,2)}.  From
that point, arithmetic functions will treat \k{x} as a 2x2 matrix. Another
attribute that can be set is the \k{class}.  This attribute can be bound to
a list of class names. For instance, \code{class(x)<-"human"}, set the class
of \k{x} to be \k{human}.  Attributes are thus used for object-oriented
programming. The S3 object system support single dispatch on the class of
the first argument of a function, whereas the S4 object system allows
multiple dispatch (on all arguments). Some of the most widely used data
type, such as data frames, leverage attributes. A data frame, for instance,
is a list of vectors with a class and a column name attribute.

\paragraph{Summary.} The most common values in R computations are vectorized
types. R programs do not have a way to constrain values to be scalar.
\k{NULL} is sometimes used to represent the case when no value is
available. \k{NA} is used within vector to represent missing observations.
Attributes can decorate values and are used as building blocks for
object-oriented programming. A potential type system for R could focus only
on the builtin types, if one wanted to strive for simplicity, or it could
try to capture attributes at the risk of increased complexity.

%
%
%
%
\subsection{R-dyntrace}
\label{sec:r-dyntrace}

\AT{Leaving this one for Aviral.}

R-dyntrace is an efficient dynamic analysis framework for R.

\subsection{contractr}
\label{sec:contractr}

\AT{I think we should move this to the Methodology section, which we should maybe rename to implementation. Thoughts?}

\contractr is an R package that implements our type system for the R
package ecosystem. It adds contracts to package functions for type-checking
arguments and results on function calls. On contract failure, it reports a type
mismatch warning with the package name, function name, argument name, argument
position, expected type signature, actual type signature of the value observed,
and a stack trace for the called function. \contractr works by modifying
function definition to insert a call to the type-checking function. When the
modified function is called, the type-checking function is invoked on the
modified function's arguments and return value.

\contractr is implemented in 434 R LOC and 2358 C++ LOC. The core
implementation is in C++ to keep the runtime overhead low. It been designed and
tested with GNU R-3.5.0 but it also works with recent versions of R. It has been
hardened with a battery of 400 test cases. We have used it extensively during
the course of this work; firstly, for sanity checking of XXX type signatures
generated for the XXX packages during the development phase, and secondly, for
assessing the quality of these type signatures on XXX packages during the
evaluation phase.

%
%
\subsubsection{Type Declarations}
Type declarations can be made available to \contractr in three ways: as
part of its internal database of types (which already has type signatures for
our corpus of 500 packages), as part of a designated \emph{TYPEDECLARATION} file
supplied by the package author and installed along with the package, or through
a user-level API function insert\_contract that lets the user insert contract to
a custom function with the type declaration provided as a string argument.

%
%
\subsubsection{Challenges}
Retrofitting a robust and efficient type-system as an external package in R has
been a significant undertaking. We had to contend with the complex design
choices of R to make it work without surprises for the end-users. We describe
two key challenges that we faced in the development of \contractr.
\begin{itemize}
\item Firstly, we faced the problem of type-checking arguments in a non-strict
  language while retaining the non-strict semantics. When a function is called in
  R, parameters are bound to unevaluated code thunks called
  promises~\cite{oopsla19}, instead of values which can be immediately
  type-checked. Naively forcing promises at function call to obtain a value for
  type-checking would violate the language semantics, leading to incorrect
  results. To preserve the original semantics, \contractr modifies the
  promise by wrapping the unevaluated promise expression in a call to its
  type-checker along with the context necessary to associate the promise to the
  function and its parameter (for failure messages). The contract checking happens
  when the promise is evaluated, either inside the function or many calls deep.
  This works, but with one wrinkle. GNU R has a bytecode compiler that can
  sometimes optimize away promises; in those cases, \contractr receives
  values that can be immediately type-checked. Non-strict semantics dictate that
  any computation related to the argument should happen on the first use of the
  argument. So type-checking at this point and issuing an error message would
  violate the non-strict semantics; if this argument is never used, it would be
  incorrect to type-check it at the beginning of call. To make this work,
  \contractr mutates the function parameter binding to a promise that wraps
  the argument value in a call to the type-checker, as in the previous case.
\item Secondly, we faced the problem of type-checking function return value. In R, the
  result of a function call is the result of evaluating the last expression in the
  function body; so an explicit return call is absent from most function
  definitions. Thus, there could be many potential sub-expressions in the function
  body which need to be type-checked against the return type. To address this,
  \contractr registers the type-checker to be called on the return value
  through a function exit hook. This hook is executed in the function call's
  environment after it has executed. While this works, we have to contend with yet
  another wrinkle, R interpreter can perform longjumps in its C implementation,
  which causes active R function calls on the stack to be discarded. When they are
  discarded, their exit hooks are called, which in this case calls the registered
  type-checker. But, these functions are in the middle of an active computation,
  so they don't have a return value to type-check. \contractr deals with
  this problem by allocating a unique sentinel object which serves as the return
  value for calls that are discarded. The function call exit hook does not call
  the type-checker if this unique value happens to be the return value of the
  call.
\end{itemize}
%
%
\subsubsection{Usability}
With \contractr, we aim to provide a hassle-free path to the R developers for
adding types to untyped package code. Hence, we have paid a lot of attention
towards usability in its design. We discuss three such usability features.
\begin{itemize}

\item \contractr enables automatic type-checking on importing without
  any user intervention. Just running \code{library(contractr)} R code is enough
  to insert contracts, both, in packages pre-loaded in the user's workspace, and,
  in packages that will be loaded eventually. To achieve this automation,
  \contractr relies heavily on R's reflective and dynamic capabilities. On
  loading, \contractr scans all the pre-loaded packages in the user's workspace
  and inserts contracts in functions for which type signatures are available. For
  all other packages installed on user's machine but not yet loaded in the
  workspace, \contractr sets up package load hooks which are executed by R when
  those packages are loaded. The hook registered by \contractr insert contracts to
  those package's functions. Thus, developers and users do not have to call the
  \contractr API to enable type-checking for packages at any point of their
  programming workflow. Furthermore, \contractr automatically removes contracts
  from all the package functions and restores them to their original state when it
  is unloaded, an operation that is rarely performed, but useful in interactive
  settings.

\item \contractr enables package authors to supply type declarations for their
  package without requiring package code modifications. Type declarations can be
  written alongside a \code{@type} section inside \roxygen function comment
  blocks. \roxygen is an R package that enables authors to add documentation to
  R functions in the form of organized plaintext sections which are
  automatically exported to R's latex style custom documentation format.
  \roxygen is a widely adopted package, used by over XXX R packages in CRAN. It
  provides an extension API for adding custom documentation tag sections in
  comment blocks. The sections are processed by \roxygen before package build
  step and a registered method for each custom tag section is invoked with the
  section data and documented object. \contractr uses this mechanism to register
  a hook for a custom \code{@type} tag. This hook parses the type declarations
  from the \code{@type} sections, extracts the function name from the documented
  function definition, and stores them in a TYPEDECLARATION file inside the
  package folder. This files is copied verbatim on package installation and
  picked up \contractr when the package is loaded. Here is an example of how a
  roxygen code block looks like with our type declaration:

  This feature enables a function and its type signature to coexist next to each
  other, where they are more likely to remain synchronized. Furthermore, the
  \contractr hook can also add the function's type declaration to its
  documentation, which seamlessly integrate our type system with the existing R
  tooling.
  
\item \contractr provides a very expressive API to the users, covering a variety
  of uses cases. During interactive development, developers can explicitly
  insert contracts by supplying the type declaration as a string argument to the
  \code{insert\_contract} API function. Conversely, contracts can be removed by
  calling the \code{remove\_contract} API on functions. Contracts can be
  selectively enabled or disabled for code blocks by wrapping them in calls to
  \code{ignore\_contracts} and \code{capture\_contracts} functions. This enables
  selective type-checking of code section during the development phase.
  Furthermore, these functions return a data frame that contains all the
  information about failed and successful contract assertions in the wrapped
  code blocks. This is useful for post-hoc investigation. Multiple type-checking
  failures can introduce a lot of noise in the program output. To alleviate this
  problem, \code{contractr.severity} option can be set to \code{'silence'}.
  This suppresses failure messages but \contractr still performs type-checking whose
  results can be explicitly obtained as a data frame from the
  \code{get\_contracts} API function. R suppresses printing of warnings on the
  terminal when a function issues too many warnings. This can hide type-checking
  failure messages. Setting \code{contractr.severity} flag to \code{'error'}
  turns the type-checking failure warnings to errors, which halts the program at
  the point of contract failure.

\end{itemize}

While \contractr's primary logic has been implemented in C++, we have not
observed a single segfault during our use of the package, either during the
development and sanity-checking phase, or during the evaluation phase.
We conclude the discussion of \contractr by drawing reader's attention to the
engineering effort it takes to package experimental ideas in the form of stable
and easy-to-use tools. R has an eclectic mix of features and design oddities
that make this even harder. 

%
%
%
%
\subsection{The Corpus}
\label{sec:corpus}

\AT{This section needs to be tight.
Reviewers were skeptical of CRAN and using the test and example code.
We should convince them otherwise.}

\AT{Some sentences that might be useful:}
We selected packages that have both \AT{a good amount of} reverse dependencies to ensure their broad usefulness, as well \AT{sufficiently high} code and branch coverage of test, example, and vignette programs to ensure that our analysis built up an accurate picture of function use.

%
%
\subsubsection{Cross-Validation Against Kaggle}

Kaggle is a website offering an environment for hosting Jupyter notebooks, and importantly provides access to a large number of open-source data sets and R and Python code.
This solved a major issue with R programs hosted on e.g. Github, where often the data is not made available.
Our aim is to run R programs hosted on Kaggle to validate the type signatures we generated for popular R packages.

\AT{Anything else to say on Kaggle, here? We will probably talk more about the corpus we analyzed later.}

%
%
%
%
%
%
\section{Methodology}

%
%
%
%
\subsection{typetracer and Initial Analysis}

\AT{So, we understand that we iterated on our type system design.
Subsequently, the analysis has undergone many iterations, but essentially it collects the most detailed information possible.
Do we want to mention all of that information, and how we ignore it later?
Or do we just want to focus on the relevant information for our final types?}

To help inform the design of our type system, and to ensure that it aligns with the day-to-day usage of R, we performed a corpus analysis of some of the most widely used R packages.
The corpus is discussed in detail in Section~\ref{sec:corpus}.
We built \typetracer, our dynamic analysis, in the R-dyntrace~\cite{oopsla19} dynamic analysis framework which is itself built on a modified version of R targeting R version 3.5.0---we refer the reader to Section~\ref{sec:r-dyntrace} for more details on the framework.

%
%
\subsubsection{Overview}
The goal of our analysis is to discover argument types for functions, builtins, and specials.
Naively intercepting function calls and recording argument types is incorrect, as recall that R wraps arguments in promises (Section \AT{what section}).
To circumvent this obstacle, we instead intercept the forcing of promises: when a promise is forced, we identify the surrounding call, collect the type of the value, and add it to the call trace associated with the current function call.
Once function execution completes, we collect information about the return value, add it to the call trace, and save the completed call trace to be written out later. 
\AT{This paragraph overlaps with the initial discussion of \contractr. We need to make sure that we don't inadvertently repeat ourselves.}

%
%
\subsubsection{Type Information}
Once a value is obtained, either on function return or when an argument promise is forced, we probe the value for the following information:

\begin{itemize}
\item the value's type tag according to R's runtime. R's runtime implements type tags as a typedef, e.g. \code{INTSXP} for integers, and, confusingly, \code{VECSXP} for lists;
\item its class, which is a list of string class names. Typically, the class of a value is held in an attribute with name ``class'', though some classes are blessed by the R runtime, such as matrix and array classes;
\item its attributes, a list of metadata attached to the value.
\end{itemize}

Now, depending on the value's type tag, we collect further information:

\begin{itemize}
\item For R's primitive types, we collect its dimension, and if it has a names attribute we collect the names as well;
\item For lists, we recursively collect the types of list elements;
\item For lists {\it with a names attribute}, we collect the names as well as the types of the slots that those names refer to (data.frames fall into this category);
\item For promises, we carefully dig into the promise value {\it without forcing it}, and collect information about the innermost value. If the innermost value is the unbound value, then the promise is an as-of-yet unevaluated expression. In this case, we do our best to guess the value of the unevaluated expression: if the expression is not a language expression, we collect information about it, otherwise we can say nothing without evaluating it and ascribe it type ``any''. \footnote{\AT{I think this is confusing, reword.}}  
\end{itemize}

For vectors (and lists), we additionally iterate through the elements, checking if the values are NA (resp. NULL); if none are NA (resp. NULL), we ascribe the NA-free (resp. NULL-free) tag to the value.
Arguments that are part of a function's varargs (denoted \ldots in function definitions) are tagged as such and ignored, as varargs can contain anything (and their type will be equivalent to and any-typed list).  

Now that we've seen the high-level picture of our approach, we will elaborate on some key details and discuss particular challenges that we faced.

%
%
\subsubsection{Challenges}
\label{subsec:typetracer-challenges}

We grappled with a number of corner cases related to R's non-strict semantics.

\begin{itemize}

\item First, we wanted to ensure that \typetracer captured type for unevaluated arguments. 
Not every argument of a function needs to be used during every function execution, and the approach we discussed above relies on argument promises being forced (and, consequently, on arguments being used).
There are a few reasons for this.

One possibility is that the argument was simply unused given the control flow of this particular execution (e.g. if the argument is only needed in one branch of an \code{if}).
This mainly occurs when arguments have default values for some parameters that are not critical to every execution oft he function.
In these cases, we still want to collect the type of the default value.

Another possibility is that a value is used {\it without being forced}, as is the case when values are ``metaprogrammed'': \AT{describe, code example.}

To tackle this, we make an {\it initial guess} of function argument types on function entry.
This collects types for argument default values, and collects an ``unused'' type for unevaluated expressions.
If an argument promise is later forced, we simply update the recorded type for the argument.

\item Second, we wanted to account for formal arguments (arguments which are named in the function definition) that were {\it not} passed to the function.
Put simply, a function was not called with all of its arguments, and no default value was specified for missing arguments.
In this case, we record a ``missing'' type for the argument.

There are two obvious ways to deal with missing arguments: type them as bot, or as top.
As we are performing a dynamic analysis, based on package test code, we conservatively type these arguments as top, or \code{any}.
It is impossible to know dynamically what all possible function behaviours are, \AT{I'm still not convinced about this.}

\item Third, we had to contend with R's C implementation sometimes performing longjumps, which discards current R calls from the call stack.
In order to ensure that call traces are not lost when a longjump occurs, \typetracer intercepts the unwinding process initiated by the longjump and mimics the functionality of the functions being exited.
The R-dyntrace framework supplies the return value during the unwinding process (if there is indeed such a value), so we can capture it, analyze it, and include it in the call trace.
In certain situations, however, no return value can be obtained (e.g. a longjump occurred as part of some error handling), and then we simply record a ``jumped'' return value, which we can deal with in post-processing.
These longjumps occur quite often, particularly when S3 dispatch occurs, as the R implementation optimizes returns from dispatched functions by initiating a longjump.

\end{itemize}

Having outlined these challenges, we will now explore the implementation of \typetracer in more detail.

%
%
\subsubsection{Select Details}

As mentioned, our analysis is built on the R-dyntrace framework.
R-dyntrace is a very efficient instrumentation framework for R, which is built on a modified version of the R runtime.
Our dynamic type analysis only needs to be run once, ever---from the analysis results, we synthesize function signatures which are passed along to later type-checking tools which do not rely on a heavily modified runtime. 

We primarily rely on 8 R-dyntrace callbacks: \texttt{closure\_entry}, \texttt{closure\_exit}, \texttt{builtin\_entry}, \texttt{builtin\_exit}, \texttt{special\_entry},\texttt{special\_exit}, \texttt{promise\_force\_entry}, and \texttt{promise\_force\_exit}.
These callbacks fire when closures, builtin, and special functions are entered and exited, and before and after the forcing of promises.
The function-related callbacks are used mainly for bookkeeping: the analysis is notified that a closure, builtin, or special has been entered/exited by pushing/popping the call onto a stack.
The calls themselves store a {\it call trace} object, where we store the type information that we collect, populated with initial guesses of argument types (as discussed in Section~\ref{subsec:typetracer-challenges}).
Further, the type of dispatch is recorded on the {\tt \_entry} variants: \AT{recall} that there are multiple ways that R performs dynamic dispatch, based on the class of function arguments.
We collect dispatch information in order to analyze dispatch usage patterns.

The main work is done in {\tt promise\_force\_exit}.
As arguments to functions in R are wrapped in promises, we delay our reflection to promise force time.
In R, promises are essentially objects with value and expression slots.
When a promise is created for a particular expression, that expression is merely put into the expression slot of a promise object, and a special {\it unbound} value is loaded into the value slot (unless some default value is specified, wherein it is loaded into the value slot---this is the case with default argument values).
When the promise is forced, the expression is evaluated in the forcing context, and the value is stored in the value slot of the promise.
So to get the type of a function argument, we check the value slot of the argument promise---perhaps recursively if the value itself is a promise.

To construct a type for each argument, we make use of R's C FFI and use low-level machinery to collect type tags and attributes from the R runtime.
The types that we provide to users are constructed during post-processing, and rely on the detailed information made available by these low-level reflection mechanisms.

%
%
%
%
\subsection{Post-Processing and Type Ascription}



%
%
%
%
%
%
\section{Type System Design}
\label{sec:typesystemdesign}

Our corpora analysis reported \AT{number} call traces for over \AT{number} functions.
We distilled these call traces into function types which captured the dynamic behaviour of the functions.
We emphasize that this was an iterative process: initial analysis suggested a certain set of types, which we tried and implemented, and further analysis suggested improvements, and so on.
In this section, we will present the final type system that we designed, which we believe captures the essence of R.
Throughout, we will touch on design decisions and weigh our choices against alternatives, speaking to the strengths and weaknesses of our chosen design.

%
%
%
%
\subsection{Basic Types}
\label{subsec:basictypes}

As discussed in Section~\ref{subsubsec:backgroundtypes}, the workhorse types in R are the vectorized primitives: integers, doubles, complex numbers, logicals (booleans), characters, and raws (bytes).
These are {\it vectorized} in that they are always considered to be vectors by the R runtime: for instance, even ``scalar'' numbers are considered to be unit-length vectors to R.
These primitive vectors are homogeneous, in that a vector of doubles contains only doubles.

Related to these primitive types is one of R's notions of ``nullness'': for each of the primitives, R distinguishes a special ``NA'' value which represents missing data.
There is an NA for each of the primitive types (i.e. there is a double NA, integer NA, etc.).
Thus, even homogeneous vectors can represent missing data with their appropriate NA value.

As with everything in R, there is nuance in even the simplest cases.
We will now discuss our design decisions relating to the vectorized primitives.

%
%
\subsubsection{Possibly-NA Primitives}

A common primitive usage pattern is related to the presence of NAs: \AT{often}, programmers explicitly check for NAs in primitive vectors, possibly sanitizing them if they are present.
For instance, consider the code in \AT{do an NA example}.
NAs complicate practically every computation using primitive vectors, a fact that many programmers appear to be acutely aware of.

%
%
\subsubsection{Scalar Primitives}

\AT{Maybe we combine the next 3 subsections into one on dimensionality? There are overlapping arguments.}
Initial analyses of our data revealed that programmers \AT{often} use scalars (as explicitly as they can, anyway).
For example, consider the code in \AT{do a scalar example}.

We found that scalars occurred more often than vectors, and chose the scalar type syntax as the default.
The syntax for primitive scalars and vectors is as follows:

\AT{syntax...}
 
%
%
\subsubsection{Forgoing Matrices}

Another layer of complexity on vectorized primitives is the presence of ``matrices'', which are \AT{more detail}.
To R, a vector is a single-dimensional matrix, and R coerces vectors to matrices where appropriate.
That said, if all vectors are matrices, then any computation that is valid on a matrix should also be valid on a vector.
Restrictions on matrix-types arguments amount to restrictions on the dimensions of the argument, which suggests a dependent type system.
We will touch on this shortly.
\AT{I can try to come up with some numbers, here. I'm open to a discussion on how best to word this.}
 
%
%
\subsubsection{Forgoing Dimensionality} 

\AT{Will revisit to tighten discussion, need to think about best way to phrase this.
Perhaps "dependent type system bad" is enough?}
Including data dimensions in types would make our type system a dependent type system, which we believe is too complex to retrofit onto a dynamic data science language such as R.
The simplest way to include dimensions would be to explicitly specify the dimensions of an argument (e.g. specifying that a function should always be called with a double vector of length 10), which is not particularly useful, and a pattern which \AT{rarely appeared in our analysis}.



%
%
%
%
\subsection{Lists}

\AT{List rehash.}
Separate from the vectorized primitives in R are lists:
Whereas vectors are homogeneous and restricted to primitive types, lists can contain arbitrarily-typed data.
R even builds on lists to create a more complex data structure known as the {\it data frame}, an essential data structure to R's functionality \AT{maybe too aggressively worded, but we can back this up with data, e.g. how many functions take data frames?}.

\AT{Outline concerns for typing.}
We are faced with several possible designs for list types.
Chief concerns include the heterogeneity of list elements, and the presence of a \code{names} attribute, allowing list elements to be accessed in a struct-like manner.
\AT{Rarely?}, programmers even write code expecting lists of a particular pre-defined length.
The next sections highlight some important design decisions for our list types.

\AT{Dig into concerns for typing.}

%
%
\subsubsection{Forgoing Tuples, or Short, Constant Length Lists}

We briefly entertained the possibility of including a \code{tuple} type, which would have looked like e.g. \code{tuple<int, int>} for a list of two integers.
When configured to detect tuples, our analysis did indeed pick up \AT{a number}, but post-processing and simplification revealed that in \AT{many} cases, they would be subsumed by co-occurring \code{list} types.
In other words, it just so happened that functions were being called with short lists.
Of the cases that tuples were not subsumed by lists, \AT{often} there were \AT{tuples of different lengths}, implying that the shape of the argument was not essential.
\AT{Wording...}

%
%
\subsubsection{Forgoing Structs}

Another pattern that we observed was the use of consistently-named lists.
In R, ``fields'' of named lists can be accessed using the \code{$} operator, such as:
\begin{lstlisting}
p <- list(x=1, y=2)
p$x + p$y
\end{lstlisting}
We thought to capture this type of behaviour with a {\it struct} type, such as \code{struct<x: integer, y: integer>}.
Initial analyses suggested that structs were very common, but further inspection revealed that they failed to tell the full story.
For instance, \AT{many} structs were merely representing list-typed values with a {\it class}, and we could capture these types more succinctly with a class-based type.
Further, it is common in R example code to use some built-in data sets, such as \AT{mention some}.
These caused undue noise, as even with our struct name unification strategy, poorly tested functions appeared to operate on e.g. \AT{funny struct type from the previous data set}.
Ultimately, we found that sticking to list types and adding a class-based was the wisest course of action.
We will discuss class types shortly.

%
%
\subsubsection{Forgoing Data Frames}

Speaking of classes, an important classful value in R is the data frame.
Data frames underpin nearly all idiomatic use of the language, and even popular \code{tibble} (from the \code{tidyverse} ecosystem) and \code{data.table} (from the \code{data.table} package) build on the essential type.
We entertained the possibility of including data frame types, but \AT{found them to be clunky? Question mark?}
Consider: \AT{example of why data.frame types would be bad atm.}

Data frames do have a type, through the \code{class<data.frame>} type, and future work will expand on this type, discussed further in Section~\ref{sec:futurework}.

%
%
%
%
\subsection{Classes in R}

As discussed in Section~\ref{sec:R}, the R runtime has more than one notion of type.
The {\it class} (rather, the {\it classes}, as values can have multiple!) of function arguments are used by R to {\it dispatch} the correct function call for the argument \AT{$\leftarrow$ wording}.
This is best served with an example:
\AT{Example, unless we talked about this earlier, in which case refer reader back.}

%
%
\subsubsection{Class Types}

The class of a value suggest a shape for the value, and we felt class to be a more succinct way of capturing struct-like types. 

\AT{We used classes to stand-in for structs.
The discussion here is naturally tied with the discussion of structs, so we need to make sure to divide the discussion appropriately.}

%
%
\subsubsection{Multi-Classes}

As R does, our type system supports multiple classes on values, and the type of a classful value is simply a list of all of its classes. 

\AT{What more to say here? Discussing occurrences of multi-classes?}

%
%
\subsubsection{Forgoing R's Object Systems}

R has multiple, disparate object systems, with the language providing direct support for S3, S4, and R5.
Thus far, we've mainly discussed S3 and its dispatch mechanism, and indeed our class types roughly capture the S3 ``types'' of values.
We will not discuss R5 as it is still under development.

For S4, we have an \code{s4} type to denote S4 values, though we did not elaborate on this type.
We stuck to simple S4 types since doing anything more would be immensely complicated, and outside of the scope of this work.
For instance, the mechanics of S4 dispatch are more complex than for S3 (S4 dispatches on the class of {\it all} arguments), and users can define their own class hierarchies that we would need to incorporate in our type analysis and contract checking frameworks.
Further, \AT{we found limited use of S4 during our analysis}, with S4 types being heavily used in a select subset of packages (\AT{such as ...}), some of which make heavy use of R's metaprogramming capabilities.
Coming up with a type system that accounts for all of these factors, and consolidates multiple object-orientation frameworks, is an interesting problem in and of itself, and we aim to extend our type system to account for this in future work.
For now, S3 types suffice: S3 is the most common object system in R, with no sophisticated hierarchies, and its semantics are simple enough to be captured by our proposed design.


%
%
%
%
%
%
\section{Evaluation Results}
\label{sec:evaluation}

\AT{The decisions we made when coming up with the types will be validated in the last section.
This section is for determining how effective the type system is.
We have two evaluation strategies currently: comparing against reverse dependencies, and comparing against the Kaggle programs.}

%
%
%
%
%
%
\section{Related Work}
\label{sec:relatedwork}

\AT{Good to mention the related projects, e.g. how this was undertaken in other languages, like Ruby.
There's a lot of literature in that space I think, I've read some of it but not all.
We also have some related work in the old paper, I'll grab it and put it here.}

%
%
%
%
%
%
\section{Conclusion}

\ldots

%
%
%
%
\subsection{Threats to Validity}

\subsection{Future Work}

\bibliography{bib/biblio,bib/jv,bib/r,bib/new,bib/gradual}

\end{document}
\appendix
\section{Appendix}



