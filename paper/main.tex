\documentclass[acmsmall,review,anonymous]{acmart}\settopmatter{printfolios=true,printccs=false,printacmref=false}
\acmJournal{PACMPL}
\acmVolume{1}
\acmNumber{OOPSLA} % CONF = POPL or ICFP or OOPSLA
\acmArticle{1}
\acmYear{2018}
\acmMonth{1}
\acmDOI{} % \acmDOI{10.1145/nnnnnnn.nnnnnnn}
\startPage{1}
\setcopyright{none}
\bibliographystyle{ACM-Reference-Format}
\citestyle{acmauthoryear}   %% For author/year citations
\usepackage{my_style}
\usepackage{listings, wrapfig,xspace}
\usepackage{paralist}
\usepackage{booktabs} % To thicken table lines

\lstset{language=R}
\definecolor{LightGray}{rgb}{.92,.92,.92}
\definecolor{Gray}{rgb}{.3,.3,.3}
\definecolor{DarkGray}{rgb}{.5,.5,.5}
\lstset{ %
  columns=flexible,
  captionpos=b,
  frame=single,
  framerule=0pt,
  tabsize=2,
  belowskip=0.5em,
  backgroundcolor=\color{LightGray},
  basicstyle=\small\ttfamily,
  emphstyle=,
  keywordstyle=,
  commentstyle=\color{Gray}\em,
  stringstyle=\color{Gray},
%  numbers=left,
  showstringspaces=false
}
\lstdefinestyle{R}{ %
  language=R,
  morekeywords={assign, delayedAssign},
  deletekeywords={env, equal, c, runif, trace, args, exp, t, all},
  breaklines=true
}
\lstdefinestyle{Rin}{ %
  style=R,
  breaklines=false
}
\renewcommand{\k}[1]{{\tt #1}\xspace}


\newcommand{\eg}{\emph{e.g.},\xspace}
\newcommand{\ie}{\emph{i.e.},\xspace}
\newcommand{\cf}{\emph{cf.}\xspace}

\newcommand{\PIR}{\textsf{PIR}\xspace}
\newcommand\pirI[1]{\mathtt{#1}}
\renewcommand{\c}[1]{{\lstinline[style=Rin]!#1!}\xspace}
\newcommand{\code}[1]{{\lstinline[style=Rin]!#1!}\xspace}
% Macros for type names from the old paper.
\newcommand{\attr}[2]{\ensuremath{#1_{\mathtt{#2}}}\xspace}
\newcommand{\attrclass}[3]{\ensuremath{#1^{\mathtt{#3}}_{\mathtt{#2}}}\xspace}
\renewcommand{\to}{\ensuremath{\rightarrow}\xspace}


\renewcommand{\null}{\texttt{\textbf{null}}\xspace}
\newcommand{\scalar}{\texttt{\textbf{scalar}}\xspace}
\renewcommand{\vector}{\texttt{\textbf{vector}}\xspace}
\newcommand{\class}{\texttt{\textbf{class}}\xspace}
\newcommand{\any}{\texttt{\textbf{any}}\xspace}
\newcommand{\environment}{\texttt{\textbf{env}}\xspace}
\newcommand{\expression}{\texttt{\textbf{expression}}\xspace}
\newcommand{\Language}{\texttt{\textbf{lang}}\xspace} %% Upper case!
\newcommand{\externalptr}{\texttt{\textbf{externalptr}}\xspace}
\renewcommand{\symbol}{\texttt{\textbf{symbol}}\xspace}
\newcommand{\pairlist}{\texttt{\textbf{pairlist}}\xspace}
\newcommand{\weakref}{\texttt{\textbf{weakref}}\xspace}
\renewcommand{\int}{\texttt{\textbf{int}}\xspace}
\newcommand{\chr}{\texttt{\textbf{chr}}\xspace}
\newcommand{\dbl}{\texttt{\textbf{dbl}}\xspace}
\newcommand{\lgl}{\texttt{\textbf{lgl}}\xspace}
\newcommand{\clx}{\texttt{\textbf{clx}}\xspace}
\newcommand{\raw}{\texttt{\textbf{raw}}\xspace}
\newcommand{\NA}{\texttt{{NA}}\xspace}
\newcommand{\VARGS}{\texttt{\textbf{...}}\xspace}
\newcommand{\T}{\ensuremath{T}\xspace}
\renewcommand{\S}{\ensuremath{S}\xspace}
\newcommand{\V}{\ensuremath{V}\xspace}
\newcommand{\A}{\ensuremath{A}\xspace}
\newcommand{\ID}{\ensuremath{ID}\xspace}
\newcommand{\F}{\ensuremath{F}\xspace}
\newcommand{\B}{\ensuremath{B}\xspace}
\newcommand{\FUN}[2]{\ensuremath{\langle #1 \rangle \rightarrow #2}\xspace}
\newcommand{\STRUCT}[1]{\texttt{\textbf{struct}}\ensuremath{\langle #1\rangle}\xspace}
\newcommand{\LIST}[1]{\texttt{\textbf{list}}\ensuremath{\langle #1\rangle}\xspace}
\newcommand{\CLASS}[1]{\texttt{\textbf{class}}\ensuremath{\langle #1\rangle}\xspace}
\newcommand{\VEC}[1]{#1\k{[]}\xspace}
\newcommand{\NAVEC}[1]{\k{\string^}\!#1\k{[]}\xspace}

\newcommand{\contractr}{{\sf ContractR}\xspace} % contractr
\newcommand{\roxygen}{{\sf Roxygen2}\xspace} % roxygen
\newcommand{\typetracer}{{\sf Typetracer}\xspace} % typetracer
\newcommand{\rdt}{{\sf R-dyntrace}\xspace}
\newcommand{\covr}{{\sf Covr}\xspace}

\newcommand{\ThresholdCodeCoverage}{65\%\xspace}
\newcommand{\ThresholdRevdeps}{5\xspace}
\newcommand{\PackageSizeOutierRnd}{5K\xspace}
\newcommand{\PackageSizeOutier}{5,000\xspace}
\newcommand{\AllCranRnd}{15.4K\xspace}
\newcommand{\AllCran}{15,373\xspace}
\newcommand{\AllInstalledRnd}{13.5K\xspace}
\newcommand{\AllInstalled}{13,514\xspace}
\newcommand{\AllLoadableRnd}{13.5K\xspace}
\newcommand{\AllLoadable}{13,506\xspace}
\newcommand{\AllWithRunnableCodeRnd}{13.4K\xspace}
\newcommand{\AllWithRunnableCode}{13,400\xspace}
\newcommand{\AllRunnableCodeRnd}{4.4M\xspace}
\newcommand{\AllRunnableCode}{4,426,364\xspace}
\newcommand{\AllExamplesCodeRnd}{1.4M\xspace}
\newcommand{\AllExamplesCode}{1,410,178\xspace}
\newcommand{\AllTestsCodeRnd}{2.6M\xspace}
\newcommand{\AllTestsCode}{2,591,456\xspace}
\newcommand{\AllVignettesCodeRnd}{424.7K\xspace}
\newcommand{\AllVignettesCode}{424,730\xspace}
\newcommand{\AllWithCoverageRnd}{11.8K\xspace}
\newcommand{\AllWithCoverage}{11,767\xspace}
\newcommand{\AllMeanExprCoverage}{60.9\%\xspace}
\newcommand{\AllMedianExprCoverage}{69.5\%\xspace}
\newcommand{\AllWithRCodeRnd}{15.3K\xspace}
\newcommand{\AllWithRCode}{15,270\xspace}
\newcommand{\AllRCodeRnd}{19.6M\xspace}
\newcommand{\AllRCode}{19,585,320\xspace}
\newcommand{\AllWithNativeCodeRnd}{3.7K\xspace}
\newcommand{\AllWithNativeCode}{3,664\xspace}
\newcommand{\AllNativeCodeRnd}{12.2M\xspace}
\newcommand{\AllNativeCode}{12,177,585\xspace}
\newcommand{\AllWAssertsRnd}{2.4K\xspace}
\newcommand{\AllWAsserts}{2,363\xspace}
\newcommand{\AllAssertsRnd}{32.3K\xspace}
\newcommand{\AllAsserts}{32,327\xspace}
\newcommand{\AllFunsWithAssertsRnd}{16.7K\xspace}
\newcommand{\AllFunsWithAsserts}{16,651\xspace}
\newcommand{\AllFunctionsRnd}{550.2K\xspace}
\newcommand{\AllFunctions}{550,222\xspace}
\newcommand{\AllPublicFunctionsRnd}{267.4K\xspace}
\newcommand{\AllPublicFunctions}{267,416\xspace}
\newcommand{\AllPrivateFunctionsRnd}{282.8K\xspace}
\newcommand{\AllPrivateFunctions}{282,806\xspace}
\newcommand{\AllMeanRevdes}{12.8\xspace}
\newcommand{\AllMedianRevdes}{2\xspace}
\newcommand{\AllRuntime}{5.7 days\xspace}
\newcommand{\AllMeanRuntime}{37.3 seconds\xspace}
\newcommand{\AllMedianRuntime}{4.9 seconds\xspace}
\newcommand{\AllUsingRoxygenRnd}{7.8K\xspace}
\newcommand{\AllUsingRoxygen}{7,790\xspace}
\newcommand{\CorpusCran}{412\xspace}
\newcommand{\CorpusInstalled}{412\xspace}
\newcommand{\CorpusLoadable}{412\xspace}
\newcommand{\CorpusWithRunnableCode}{412\xspace}
\newcommand{\CorpusRunnableCodeRnd}{385.8K\xspace}
\newcommand{\CorpusRunnableCode}{385,766\xspace}
\newcommand{\CorpusExamplesCodeRnd}{98.9K\xspace}
\newcommand{\CorpusExamplesCode}{98,881\xspace}
\newcommand{\CorpusTestsCodeRnd}{258K\xspace}
\newcommand{\CorpusTestsCode}{258,006\xspace}
\newcommand{\CorpusVignettesCodeRnd}{28.9K\xspace}
\newcommand{\CorpusVignettesCode}{28,879\xspace}
\newcommand{\CorpusWithCoverage}{412\xspace}
\newcommand{\CorpusMeanExprCoverage}{80.8\%\xspace}
\newcommand{\CorpusMedianExprCoverage}{80\%\xspace}
\newcommand{\CorpusWithRCode}{412\xspace}
\newcommand{\CorpusRCodeRnd}{760.6K\xspace}
\newcommand{\CorpusRCode}{760,553\xspace}
\newcommand{\CorpusWithNativeCode}{189\xspace}
\newcommand{\CorpusNativeCodeRnd}{534.4K\xspace}
\newcommand{\CorpusNativeCode}{534,362\xspace}
\newcommand{\CorpusWAsserts}{153\xspace}
\newcommand{\CorpusAssertsRnd}{2K\xspace}
\newcommand{\CorpusAsserts}{1,995\xspace}
\newcommand{\CorpusFunsWithAssertsRnd}{1.3K\xspace}
\newcommand{\CorpusFunsWithAsserts}{1,306\xspace}
\newcommand{\CorpusFunctionsRnd}{38.2K\xspace}
\newcommand{\CorpusFunctions}{38,167\xspace}
\newcommand{\CorpusPublicFunctionsRnd}{17.4K\xspace}
\newcommand{\CorpusPublicFunctions}{17,354\xspace}
\newcommand{\CorpusPrivateFunctionsRnd}{20.8K\xspace}
\newcommand{\CorpusPrivateFunctions}{20,813\xspace}
\newcommand{\CorpusMeanRevdes}{45.5\xspace}
\newcommand{\CorpusMedianRevdes}{12\xspace}
\newcommand{\CorpusRuntime}{2.9 hours\xspace}
\newcommand{\CorpusMeanRuntime}{25.8 seconds\xspace}
\newcommand{\CorpusMedianRuntime}{7.3 seconds\xspace}
\newcommand{\CorpusUsingRoxygen}{180\xspace}

\newcommand {\PercUnitypedPosition} {87.2\%\xspace} %% 0.8717
\newcommand {\PercManytypedPosition} {2\%\xspace} %% 0.018
\newcommand {\PercUnitypedPositions} {87.2\%\xspace} %% 0.8717
\newcommand {\PercManytypedPositions} {2\%\xspace} %% 0.018

\newcommand{\CranAssertsRnd}{32.3K\xspace}
\newcommand{\CranAsserts}{32,327\xspace}
\newcommand{\CranAssertsInPackagesRnd}{2.4K\xspace}
\newcommand{\CranAssertsInPackages}{2,363\xspace}
\newcommand{\CranAssertsInFunctionsRnd}{15.9K\xspace}
\newcommand{\CranAssertsInFunctions}{15,929\xspace}
\newcommand{\CranStopifnotRatio}{87.6\%\xspace}
\newcommand{\CranTypedAssertsRnd}{12K\xspace}
\newcommand{\CranTypedAsserts}{11,997\xspace}
\newcommand{\CranTypedAssertsRatio}{37.1\%\xspace}
\newcommand{\CranTypedAssertsPackagesRnd}{1.5K\xspace}
\newcommand{\CranTypedAssertsPackages}{1,487\xspace}
\newcommand{\CranTypedAssertsPackagesRatio}{62.9\%\xspace}
\newcommand{\CranTypedAssertsFunctionsRnd}{7.1K\xspace}
\newcommand{\CranTypedAssertsFunctions}{7,119\xspace}
\newcommand{\CranTypedAssertsFunctionsRation}{44.7\%\xspace}
\newcommand{\CranPartiallyTypedAssertsRnd}{15.5K\xspace}
\newcommand{\CranPartiallyTypedAsserts}{15,487\xspace}
\newcommand{\CranPartiallyTypedAssertsRatio}{47.9\%\xspace}
\newcommand{\CranPartiallyTypedAssertsPackagesRnd}{1.7K\xspace}
\newcommand{\CranPartiallyTypedAssertsPackages}{1,664\xspace}
\newcommand{\CranPartiallyTypedAssertsPackagesRatio}{70.4\%\xspace}
\newcommand{\CranPartiallyTypedAssertsFunctionsRnd}{9K\xspace}
\newcommand{\CranPartiallyTypedAssertsFunctions}{9,045\xspace}
\newcommand{\CranPartiallyTypedAssertsFunctionsRation}{56.8\%\xspace}
\newcommand{\CorpusAssertsRnd}{2K\xspace}
\newcommand{\CorpusAsserts}{1,995\xspace}
\newcommand{\CorpusAssertsInPackages}{153\xspace}
\newcommand{\CorpusAssertsInFunctionsRnd}{1.3K\xspace}
\newcommand{\CorpusAssertsInFunctions}{1,264\xspace}
\newcommand{\CorpusStopifnotRatio}{94.7\%\xspace}
\newcommand{\CorpusTypedAssertsRnd}{1K\xspace}
\newcommand{\CorpusTypedAsserts}{1,005\xspace}
\newcommand{\CorpusTypedAssertsRatio}{50.4\%\xspace}
\newcommand{\CorpusTypedAssertsPackages}{114\xspace}
\newcommand{\CorpusTypedAssertsPackagesRatio}{74.5\%\xspace}
\newcommand{\CorpusTypedAssertsFunctions}{688\xspace}
\newcommand{\CorpusTypedAssertsFunctionsRation}{54.4\%\xspace}
\newcommand{\CorpusPartiallyTypedAssertsRnd}{1.2K\xspace}
\newcommand{\CorpusPartiallyTypedAsserts}{1,223\xspace}
\newcommand{\CorpusPartiallyTypedAssertsRatio}{61.3\%\xspace}
\newcommand{\CorpusPartiallyTypedAssertsPackages}{125\xspace}
\newcommand{\CorpusPartiallyTypedAssertsPackagesRatio}{81.7\%\xspace}
\newcommand{\CorpusPartiallyTypedAssertsFunctions}{859\xspace}
\newcommand{\CorpusPartiallyTypedAssertsFunctionsRation}{68\%\xspace}
\newcommand{\AssertthatRevdeps}{211\xspace}
\newcommand{\AssertrRevdeps}{2\xspace}


\begin{document}
\title{Designing Types for R, Empirically}

\newcommand{\NUMFUNCTIONS}{20,214\xspace}  %%% TODO auto-gen
\newcommand{\NUMPACKAGES}{412\xspace}  %%% TODO auto-gen
\newcommand{\PACKAGES}{412\xspace}  %%% TODO auto-gen
\newcommand{\genthat}{genthat\xspace}  %%% TODO auto-gen
\newcommand{\YEARS}{20\xspace} %%TODO fix
\newcommand{\PERCFAILEDASSERTIONS}{2.19\%\xspace}
\newcommand{\PERCFAILEDASSERTIONSWUNDEF}{2.23\%\xspace}
\newcommand{\PERCASSERTIONSUNDEF}{0.04\%\xspace}
\newcommand{\NUMPKGSEVAL}{7485\xspace} % autogenerate
\newcommand{\NUMASSERTIONS}{62,171,573\xspace} %autogenerate
\newcommand{\PROPFUNSFAILEDCHECK}{21.21\%\xspace}
\newcommand{\PROPFUNSFAILEDCHECKNOSTHREE}{15.48\%\xspace} 
\newcommand{\PROPARGSFAILINGASSERTS}{3.53\%\xspace} % autogen....
\newcommand{\PERCSUCCARG}{96.46\%\xspace}
\newcommand{\PERCSUCCFUNNOSTHREE}{84.52\%\xspace}
\newcommand{\PERCSUCCFUNS}{78.79\%\xspace}
\newcommand{\PERCCALLSBADFUNSINTESTS}{5.38\%\xspace}
\newcommand{\TOTNUMSIGS}{21,730\xspace}
\newcommand{\PROPSTRUCTSDYN}{44.46\%\xspace}
\newcommand{\NUMFILESRUN}{145,373\xspace}


\begin{abstract}
The R programming language is widely used in a variety of domains for tasks
related to data science. R was designed to favor an interactive style of
programming with minimal syntactic and conceptual overhead. This design is
well suited to interactive data analysis, but a bad fit for tools such as
compilers or program analyzers which must generate performant code or catch
programming errors.  In particular, R has no type annotations, its
operations are dynamically checked at run-time. The starting point for our
work are the twin questions, \emph{what expressive power is needed to
  accurately type R code?} and \emph{which type system is the R community
  willing to adopt?} Both questions are difficult to answer without actually
designing a type system and attempting to convince users and package
developers to try the proposed system.  The goal of this paper is to provide
data that can feed into the design process. To this end, we perform a large
corpus analysis with aim to gain insights in the degree of polymorphism
exhibited by idiomatic R code and explore potential benefits that the R
community could accrue from even a simple type system.  As a starting point,
we infer type signatures for \NUMFUNCTIONS functions from \NUMPACKAGES
packages among the most widely used open source R libraries. The signatures
use a minimal type language with only basic types, vectors and classes.  We
implement a contract checker that can be used to validate the inferred type
signatures.
\end{abstract} \maketitle

\section{Introduction}

Our community builds, improves, and reasons about programming languages.  To
make design decisions that benefit users, we need to understand our target
language as well as the real-world needs it answers. Often, we can appeal to
our intuition, as many languages are intended for general purpose
programming tasks. Unfortunately, intuition may fail when looking at
domain-specific languages as these languages designed for a specific group
of users to solve very specific needs. This is the case of the data science
language R.

R and its ancestor S were designed, implemented, and maintained by
statisticians. Originally they aimed to be glue languages for reading data
and calling statistical routines written in Fortran. Over three decades they
became widely used across many fields for data analysis and visualization.
Modern R, as an object of study, is fascinating. It is a vectorized,
dynamically typed, lazy functional language with limited side-effects,
extensive reflective facilities and retrofitted object-oriented programming
support.

Many of the design decisions that gave us R were intended to foster an
interactive, exploratory, programming style. These include, to name a few,
the lack of type annotations, the ability to use syntactic shortcut, and the
automatic conversion between data types.  While these choices have led to a
language with a low barrier to entry---many data science educational
programs do not teach R itself but simply introduce some of its key
libraries---they have also created a language where errors can go
undetected.

Retrofitting a type system to the R programming language would increase our
assurance in the result of data analysis. But, we are faced with two
challenges. First, it is unclear what would be the \emph{right} type system
for a language as baroque as R. For example, one of the most popular data
type, the \code{data.frame}, is manipulated through reflective operations --
a data frame is a table whose columns can be added or removed on the fly.
Second, but just as crucially, designing a type system that will be adopted
would require overcoming some prejudices and educating large numbers of
users.

This paper is a data-driven study of what a type system for the R language
could look like. Our intention is to eventually propose changes to the
language, but we are aware that for any changes to be accepted by the user
community they must clearly benefit the language without endangering
backwards compatibility. Our goal is thus to find a compromise between
simplicity and usefulness; any proposed type system should cover common
programming idioms while remaining easy to learn and to use.

This paper focuses on a simpler problem than an entire type system, instead,
we limit the scope of our investigation to giving types to function
signatures. In order to do this, we design a simple type language, one that
matches the R data types but omits features such as parametric polymorphism
and subtyping between user-defined data types. We then extract type
signature from execution traces of a corpus of widely used libraries.  This
allows us to see how far one can get with the simple type language and
identify limitations of our this particular design. We validate the
robustness of the extracted type signature by implementing a contact system
that weaves the types around their respective functions, and use a large
number of clients of the target packages for validation. The contract system
can be used by the R community to experiment with our type signatures and to
replace ad hoc error checking code.

To sum up, our paper makes the following main contributions:
\begin{itemize}
\item We implemented tooling to automatically extract type signatures from R
  functions and to instrument R functions with checks based on their declared
  types. The tooling is robust and scalable to the entire R language.
\item We carried out a large-scale analysis of corpus of \PACKAGES widely
  used and actively maintained libraries to extract function type signatures and
  validated the robustness of the inferred type signatures against \NUMFILESRUN
  programs that use those functions.
\item We report on the appropriateness and usefulness of a simple type
  language for the R programming language.
\end{itemize}

Our tools are open source and publicly available on GitHub (link anonymize),
all results in this paper are reproducible and will be submitted for
artifact evaluation should this paper be accepted.
%
\section{Background} %%%%%%%%%%%%%%%%%%%%%%%%%%%%%%%%%%%%%%%%%%%%%%%%%%%%%%%%%

In this section, we will introduce related work and the R programming language.

\subsection{Related Work}

Dynamic programming languages such as Racket, JavaScript and PHP have been
extended post-facto with various static type systems.  In each case, the
type system was carefully engineered to match the salient characteristics of
the host languages and to foster a particular programming style. For
example, Typed Racket emphasizes functional programming and support the
migration from untyped to fully typed code~\cite{tf-popl08},
Hack~\cite{hack13} and TypeScript~\cite{BAT14} focus on typing
object-oriented features of PHP and JavaScript, respectively. They allow to
intersperse typed and untyped code in a fine-grained manner.

But what if the design of the type system is unclear?  \citet{tip} propose
an intriguing approach called trace typing. With trace typing, a new type
system can be prototyped and evaluated by applying the type rule to
execution traces of programs. While the approach has the limitation of
dynamic analysis techniques, namely that the results are only as good at the
coverage of the source code, it allows to quickly test new design and
quantify how much of a code base can be type-checked.  Other approaches that
infer types for dynamic analysis include the work of \citet{FurrAF2009} for
Ruby.

There is no previous work for the R programming language. We take
inspiration in the above mentioned works but focus on adapting them to our
target language.


\subsection{The R Programming Language} %%%%%%%%%%%%%%%%%%%%%%%%%%%%%%%%%

The R Project is a key tool for data analysis.  At the heart of the R
project is a \emph{vectorized, dynamic, lazy, functional, object-oriented}
programming language with a rather unusual combination of
features~\cite{ecoop12} designed to ease learning by non-programmer and
enable rapid development of new statistical methods.  The language was
designed~\citet{R96} as a successor to S~\cite{S88}.

In this paper, we will attempt infer type signature for functions. We now
introduce some relevant concepts.  Functions can be called with named
parameters, arguments can have default values, and R also support variable
argument lists.  To illustrate all of this in a single example, consider:

\begin{lstlisting}
  f <- function(x, ..., y=if(z==0) 1, z=0) x + y + if (missing(...)) 0 else c(...)
\end{lstlisting}

\noindent
This function four formal parameters, \k x, \k{...}, \k y and \k z. Argument
\k x can be bound positionally or passed by name. The vararg argument,
\k{...}, is always passed positionally. The remaining two arguments must be
passed by name.  Arguments \k y and \k z have default values, in the case of
\k z this is a constant, but \k{y}'s default value is an expression that
depends on the value of \k z. The body of the function will add parameters
\k x and \k y with either the scalar 0 or the result of concatenating the
varargs into a primitive vector. The function \k{missing} tests if an
argument was explicitly passed. The following are some valid invocations
of \k f:

\begin{lstlisting}
 > f(1)    
 [1] 2             % a double vector, y is 0, ... is missing
 > f(1,2) 
 [1] 4             % a double vector, y is 0, ... is 2
 > f(1,2,3)
 [1] 4 5           % a double vector, y is 0, ... is 2, 3
 > f(2,3,x=1)
 [1] 4 5           % a double vector, y is 0, ... is 2, 3
 > f(x=1, y= 1)
 [1] 2             % a double vector, y is 1, ... is missing
 > f(x=1, z= 1)
 numeric(0)        % a double vector of length 0, y is NULL
 > f(1L,2L,y=1L)
 [1] 4             % an integer vector, y is integer 1, ... is integer 2
 > f(1, y=c(1,2))
 [1] 1 2           % a double vetor, y is 1, 2, ... is missing
\end{lstlisting}

\noindent
The above hints at the polymorphism of the language, \k f, may return a
vector of integer or of double, the length of the vector depends on the
length of the varargs and of \k x and \k y.  The language does not really
differentiate between scalar and vectors.
Some more exotic types that can be encountered include vectors of complex
number and list of arbitrary types. Function \k f can be invoked with those
as well.

\begin{lstlisting}
 > c1 <- complex(re=1, im=2)
 > c2 <- complex(re=2, im=1)
 > f(c1)
 [1] 1             % a double vector, x is complex, y is double
 > f(c1,y=c2)
 [1] 2+1i          % a complex vector, x and y are complex
 > l1 <- list(1,2)
 > l2 <- list(c1,c2)
 > f(l1) 
 [1] 1             % a double vector, x is a list of doubles
 > f(l1,y=l2)
 [[1]]             % a list of complex, x is a list of doubles
 [1] 1+2i          %                      y is a list of complex
 
 [[2]]
 [1] 2+1i
\end{lstlisting}

R has one builtin notion of type that can be queried by the \k{typeof}
function. Figure~\ref{rtypes} lists all of the builtin types that are
provided by the language. They are the possible return values of
\k{typeof}. There is no intrinsic notion of subtyping in R. But, in many
context a \k{logical} will convert to \k{integer}, and an \k{integer} will
convert to \k{double}.  Some odd conversions can occur in corner cases, such
as \k{1<"2"} holds and \k{c(1,2)[1.6]} returns the first element of the
vector, as the double is converted to an integer. R does not distinguish
between scalars and vectors (they are all vectors), so \code{typeof(5) ==}
\code{typeof(c(5)) == typeof(c(5,5))} \code{ == "double"}. Finally all
vectorized data types have a distinguished missing value denoted by
\code{NA}. The default type of \code{NA} is \k{logical}. We can see that
\code{typeof(NA)=="logical"}, but NA inhabits every type:
\code{typeof(c(1,NA)[2])=="double"}.

\begin{wrapfigure}{r}{6.1cm}
\footnotesize\begin{tabular}{l|l@{}}\hline
\multicolumn{2}{l}{\bf Vectorized data types:}  \\\hline
\k{logical}   & vector of boolean values\\
\k{integer}   & vector of 32 bit integer values\\
\k{double}    & vector of 64 bit floating points\\
\k{complex}   & vector of complex values\\
\k{character} & vector of strings values\\
\k{raw}       & vector of bytes\\
\k{list}      & vector of values of any type\\\hline
\multicolumn{2}{l}{\bf Scalar data types:}\\\hline
\k{NULL}      &  singleton null value\\
\k{S4}        &  instance of a S4 class \\
\k{closure}   &  function with its environment\\
\k{environment}& mapping from symbol to value \\\hline
\multicolumn{2}{l}{\bf Implementation data types:}\\\hline
\multicolumn{2}{l}{\k{special},
\k{builtin},
\k{symbol},
\k{pairlist},
\k{promise}}\\
\multicolumn{2}{l}{
\k{language},
\k{char},
\k{...},
\k{any},
\k{expression},
}\\
\multicolumn{2}{l}{
\k{externalprt},
\k{bytecode},
\k{weakref}}\\\hline
\end{tabular}\caption{Builtin Types}\label{rtypes}\end{wrapfigure}

With one exception all vectorized data types are monomorphic, the exception
is the \k{list} type which can hold values of any other type including
\k{list}. For all monomorphic data types, attempting to store a value of a
different type will cause a conversion. Either the value is converted to the
type of vector, or the vector is converted to the type of the value.

Over the years, programmers have found the need for a richer type structure
and have added {\it attributes}. The best way to think of attributes is as
an optional map from name to values that can be attached to any object.
Attributes are used to encode various type structures. They can be queried
with functions such as \k{attributes} and \k{class}.  The addition of
attributes lets programmers extend the set of types by tagging data with
user-defined attributes. For example, one could define a vector of four
values, \code{x<-c(1,2,3,4)} and then attach the attribute \k{dim} with a
pair of numbers as value: \code{attr(x,"dim")<-c(2,2)}.  From that point,
arithmetic functions will treat \k{x} as a 2x2 matrix. Another attribute
that can be set is the \k{class}.  This attribute can be bound to a list of
class names. For instance, \code{class(x)<-"human"}, set the class of \k{x}
to be \k{human}.  Attributes are thus used for object-oriented
programming. The S3 object system support single dispatch on the class of
the first argument of a function, whereas the S4 object system allows
multiple dispatch (on all arguments). Some of the most widely used data
type, such as data frames, leverage attributes. A data frame, for instance,
is a list of vectors with a class and a column name attribute.

Scalar data types include the distinguished \k{NULL} value, which is also of
type \k{NULL}, instance of classes written using the S4 object system,
closures and environments.  The implementation of R has a number of other
types listed in Figure~\ref{rtypes} for reference.

%%%%%%%%%%%%%%%%%%%%%%%%%%%%%%%%%%%%%%%%%%%%%%%%%%%%%%%%%%%%%%%%%%%%%%%%%%%%%%
\section{Designing A Type Language for R}%%%%%%%%%%%%%%%%%%%%%%%%%%%%%%%%%%%%%


This section sets out to propose a candidate design for a type language that
can be used to describe the arguments and return values of functions in R.
The goal is not to be the final design but rather a starting point for an
iterative process.

Fig.~\ref{types} presents our type language. An early design choice was to
stay close to the R language and only depart in small, hopefully,
non-controversial ways.  Functions types have the form \FUN{A_1, \dots,
  A_n}\T where each $\A_i$ argument is either a type \T or \VARGS, a
variable length argument list. The rest of this section details and
motivates our design.

\begin{figure}[!h] \noindent \small  \centering\begin{minipage}{.45\linewidth}
\begin{tabular}{lclr}
\T& ::= & \any          & \it top type\\
  & |   & \null         & \it null type\\
  & |   & \environment  & \it environment type \\
  & |   & \S            & \it scalar type \\
  & |   & \V            & \it vector type \\
  & |   & \T~\k{|}\T    & \it union type \\
  & |   & \k{?} \T      & \it nullable type \\
  & |   & \FUN{A_1, ..., A_n}\T    & \it function type \\
  & |   & \LIST\T                  & \it list type\\
  & |   & \CLASS{ID_1, ..., ID_n}  & \it class type\vspace{5pt}\\
\end{tabular}\end{minipage}   \hfill
\begin{minipage}{.45\linewidth}\begin{tabular}{lclr}
\A & ::= & \T          & \it arguments \\
   & |   & \VARGS      & \vspace{5pt} \\
\V & ::= & ~\VEC\S     & \\
   & |   & \NAVEC\S    & \it na vector types \vspace{5pt}\\
\S & ::= & \int        & \\
   & |   & \chr \\
   & |   & \dbl \\
   & |   & \lgl \\
   & |   & \clx \\
   & |   & \raw \vspace{5pt}\\
\end{tabular}\end{minipage}\caption{The R type language}\label{types}
\end{figure}

\paragraph{Scalar vs. Vectors} Our type system distinguishes between
vectors of length 1, and vectors of any dimensions. The six base types,
integers, doubles, complex numbers, logicals (booleans), characters, and
raws (bytes), can be either vectors (e.g., \VEC\int) or scalars (e.g.,
\int).  It is worthwhile to note that a vector can happen to be of length 1
and thus a scalar value is a subtype of the corresponding vector type.
Vectors are homogeneous, in that a vector of doubles contains only doubles.

\paragraph{Missing values} In R, a missing value is written
\NA, each of basic types has its specific representation of \NA. Thus there
is an \int missing observation as well as a double missing, and they are
different.  The type system allows to distinguish between vectors that may
contain missing observations (written \NAVEC\int) and those who are
guarantee to be \NA-free (\VEC\int).  A \NA-free vector can be treated as a
vector that may have missing observations.  We do not allow scalar values to
be missings.  The type \raw does not allow {\NA}s.

\paragraph{Unions} We support untagged unions of types written
$\T_1 \k{|} \T_n$.

\paragraph{Nullables} 
The type system has a \null type that is inhabited by a singleton
\code{NULL} value. This value is often used as a sentinel or as a stand in
for arguments that were not provided.  To capture this behavior, we
introduce a nullable type \code{?}\T.  The difference between the \NA and
\code{NULL} values is that \NA can be stored in primitive vectors.

\paragraph{Lists} Heterogeneous collections are implemented using lists.
Lists and vectors are closely related. Any vector can be converted to a list
with \code{as.list}, and lists to vectors with \code{unlist} (coercions may
ensue). We support parametrization of the list type, \LIST\T, but for
heterogeneous lists this type will default to \any.

\paragraph{Classes} 
R has more than one notion of type. Values can be attributed, one important
attribute is the class of a value.  A class is a list of names that are used
to mimic object-oriented programming. The type system includes class types
written \CLASS{\ID_1,\dots,\ID_n}.

\paragraph{Environments}
Environments are lists with reference semantics: mutating a value in an
environment is performed in-place.  They are used to store variables and to
escape from the copy-on-write semantics of other data types.  Environments
are comprised of name, value pairs, and each environment has a reference to
its parent environment. 

%
%
\section{Extracting and Checking Signatures}

For this paper, we have built tooling to (\emph{a}) automate the extraction
of raw type signatures from execution traces, (\emph{b}) infer type
signatures from a set of raw types, and (\emph{c}) validate the inferred
signatures by the means of contracts. This section presents each step
of this pipeline.

\subsection{Extracting raw type signature from traces}

We implemented \typetracer, an automated tool for extracting raw type
signatures from execution traces of R programs. The goal of this tool is to
output a tuple $\langle f, t_1, \dots, t_n, t\rangle$ for each function call
during the execution of a program, where $f$ is an identifier for a
function, $t_i$ are type-level summaries of the arguments and $t$ is a
summary of the return value.

While this task is seemingly simple, the details and their proverbial devil
are surprisingly tricky to get right and to scale to long running programs.
To avoid starting from scratch, our implementation reuses an open source
dynamic analysis framework for R named \rdt~\cite{oopsla19} which consists
of an instrumented R Virtual Machine based on GNU-R version 3.5.0. The
framework exposes hooks at key points in the R interpreter to which user
defined callbacks can be attached. These hooks include function entry and
exit, method dispatch for the S3 and S4 object systems, the longjumps used
by the interpreter to implement non-local exit, creation and forcing of
promises (lazy evaluation), variable definition, value creation, mutation
and garbage collection.

\paragraph{Raw types}
The type information output by the tool includes the \emph{type tag} of each
value. The R internal types are translated to names in the proposed type
system. The next bit of information output is the \emph{class} which is an
optional list of names, it may be absent, and, in some cases, it may be
implicit (i.e. the interpreter blesses some values with the \k{matrix} and
\k{array} classes even when they have no attributes). 
Depending on the value's type tag, the tool collects further information:

\begin{itemize}
\item For vectors, the number of dimensions and the present of \NA values.
\item For lists, a recursive traversal collects element types.
\item For promises, we attempt to establish the type of its result or output
  \any.
\end{itemize}

\noindent
To obtain raw types, we make use of R's C FFI and use low-level machinery to
collect type tags and attributes from the R run-time.  The types that we
provide to users are constructed during post-processing, and rely on the
detailed information made available by these low-level reflection
mechanisms.

\paragraph{Varargs}
Arguments that are part of a function's varargs (denoted \VARGS) are
ignored, as type \any is output for the entire varag construct.


\paragraph{Promises}
The fact that arguments are evaluated lazily (in that expression are packed
into promises and evaluated on first access) complicates the information
gathering.  For example, some promises remain un-evaluated, and it would be
erroneous to force them as they may cause side-effects that will affect the
programs execution.  To deal with unevaluated arguments, we make an initial
guess for each argument at function entry.  If the promise is later forced,
we simply update the recorded type for the argument.

\paragraph{Missing arguments}
Parameters which receive no values when the function is called are termed
missing. This occurs when a function was called with too few arguments and
no default value was specified for missing arguments.  We record a
\k{missing} type for such argument.  There are two obvious ways to deal with
missing arguments: type them as \any or type them as some unit type.  As we
are performing a dynamic analysis, we conservatively type them as
\LIST{\any}.

\paragraph{Non-local returns}
When a function exits with a longjump there is no return value to speak
of. In order to ensure that call traces are valid when a longjump occurs, we
intercept the unwinding process initiated by the longjump and mimic the
functionality of the functions being exited. When no return value is present
for a call, we record a special \k{jumped} return type.

\paragraph{Implementation details.}
We primarily rely on eight callbacks: \k{closure\_entry}, \k{closure\_exit},
\k{builtin\_entry}, \k{builtin\_exit}, \k{special\_entry},
\k{special\_exit}, \k{promise\_force\_entry}, and
\k{prom\-ise\_force\_exit}.  The function-related callbacks are used mainly
for bookkeeping: the analysis is notified that a construct has been entered
by pushing the call onto a stack.  The calls themselves store a trace
object, it is that object that holds the type information. As R can perform
single or multiple dispatch on function arguments depending on their class,
the relevant information is kept by the {\tt \_entry} variants.



\subsection{Inferring type signatures from raw types}

The output of the \typetracer tool consists of a set of tuples of raw types,
each representing a function call in some program's execution. The inference
step consolidates the different tuples corresponding to a particular function
definition across multiple programs and distills them into a single type
signature.  The shape of the inferred function signatures is:

\medskip
\FUN{ T_{1,1} \k{~|~} T_{1,i}, \dots, T_{n,1} \k{~|~} T_{n,j}}{T_1\k{~|~} T_k}
\medskip

\noindent
In other words, we take the union of the types occurring at individual
argument positions rather than an union of function types. Furthermore, we
apply some transformation on the types to keep the size of types in check.

\medskip
\begin{tabular}{rclll}
  \T \k{~|~} \T  & $\Rightarrow$ &  \T &  \\
  \T \k{~|~} $T'$ & $\Rightarrow$ &  \T & \it{iff} $T' <: T$  \\
  \LIST{\T} \k{~|~} $\LIST{T'}$ & $\Rightarrow$ &  \LIST\T & \it{iff} $T' <: T$  \\
   $\null \k{~|~} \VEC{\S_1} \k{~|~}\dots \VEC{\S_n}$  & $\Rightarrow$ &
    $ \NAVEC{\S_1} \k{~|~}\dots \NAVEC{\S_n}$ &  \\  
\end{tabular}
\medskip

\noindent
Assuming that type sequences can be reordered freely, we rewrite types to
minimize their size by removing redundant types, types that are subsumed by
subtyping, and \null types. Lists with more than 5 types in their union become \LIST{any}.
We also only generate types for the first 20 arguments of a function.

\subsection{Checking types signatures with contracts}

One way to validate the inferred type annotations is to check that different
programs using one of the functions for which a type signature has been
inferred respect the signature.  \contractr is an R package that allows to
decorate function with assertions. We use it to insert type checks of the
arguments and return values. As usual with R, this is not straightforward.

While \contractr's primary logic has been implemented in C++ to reduce its
runtier overhead, we have not observed a single segfault during our use of
the package.  It has been tested with GNU R-3.5.0 and hardened with a
battery of 400 test cases. It works by modifying function definition to
insert a call to a type-checking function.

In terms of usability, our checker is enabled automatically whenever a
package is loaded in R. So, a simple invocation of \code{library(contractr)}
causes contracts to be injected in all packages.  On loading, \contractr
scans all packages in the user's workspace and inserts contracts in
functions for which type signatures are available. Then \contractr sets up
package load hooks which are executed when new packages are loaded.
Furthermore, \contractr automatically removes contracts from all functions
and restores them to their original state when it is unloaded.

The type signatures can be provided externally, in a file, thus avoiding the
need to change the source code of checked packages.  Type declarations can
also be written alongside a \code{@type} section inside \roxygen function
comment blocks. \roxygen is an R package that enables authors to add
documentation. The documentation sections are processed by \roxygen before
package build and a registered method for each custom tag section is
invoked with the section data and documented object. \contractr uses this
mechanism to register a hook for a custom \code{@type} tag. 
  
The \contractr API allows users to explicitly insert contracts during
interactive development by supplying the type signature as a string.
Contracts can be removed by calling \code{remove\_contract}.  Contracts can
be selectively enabled or disabled for code blocks by wrapping them in calls
to \code{ignore\_contracts} and \code{capture\_contracts}.  These functions
return a data frame that contains all the information about failed and
successful contract assertions in the wrapped code blocks.  Multiple
failures can introduce noise; this is avoided by setting
\code{contractr.severity} to \code{'silence'}.  \contractr still performs
type-checking so that results can be obtained by calling
\code{get\_contracts}. Setting \code{contractr.severity} flag to
\code{'error'} turns the type-checking warnings to errors and
halt the program.

\paragraph{Checking the values} A number of properties can be checked
by a simple tag check, namely whether a value is a: null, environment, vector
or scalar. Other properties require an inspection of the contents of a
value: the absence of \NA or the type of the elements of list. Union types
require checking for each of the alternatives. We do not check higher-order
functions in this version of the tool.

When argument values are wrapped in promises (this is not always the case
due to compiler optimizations), in order to retain the non-strict semantics
of R, the expression held in the promise is wrapped in a call to the type
checker, and type checking is delayed until the promise is forced. This
leads to corner cases such that the type checking of a function may happen
after that function has returned.

Return values require care as well.  Functions return the last expression
they evaluate, thus to avoid having to analyze the code of the called
function the checker will register a callback on the exit hook. The hook is
executed in the function call's environment. Another wrinkle is due to
longjumps which causes active function calls on the stack to be
discarded. When they are discarded, their exit hooks are called.  But, these
function do not have a return value to type-check. \contractr deals with
this problem by allocating a unique sentinel object which serves as the
return value for calls that are discarded. The exit hook does not call the
type-checker if it see the sentinel.

%%%%%%%%%%%%%%%%%%%%%%%%%%%%%%%%%%%%%%%%%%%%%%%%%%%%%%%%%%%%%%%%%%%%%%%%%%%%%
\section{Project Corpus}\label{sec:corpus} %%%%%%%%%%%%%%%%%%%%%%%%%%%%%%%%%%

For this paper we have selected \CorpusLoadable packages consisting of
\CorpusRCodeRnd lines of R code and \CorpusNativeCodeRnd lines of native code
(C/C++/Fortran). Figure~\ref{fig:corpus} shows these packages, the size of the
dots reflects the size of the project in terms of lines of code including both R
and native code\footnote{ Lines of source code reported excludes comments and
  blank lines, counted by \emph{cloc}, \cf
  \url{https://github.com/AlDanial/cloc}}, the x-axis indicates the expression
code coverage in percents and the y-axis gives the number of reverse
dependencies in log scale. Dotted lines indicate means. Packages with over
\PackageSizeOutierRnd lines of code are annotated.

\begin{figure}[!h]  \centering
  \includegraphics[width=.8\linewidth]{plots/corpus.pdf}
  \caption{Corpus overview}\label{fig:corpus}
\end{figure}

These packages come from the Comprehensive R Archive Network
(CRAN\footnote{\url{http://cran.r-project.org}}), the largest repository of R
code with over \AllCranRnd packages\footnote{CRAN receives about 6 new package
  submissions a day~\cite{Ligges2017}} containing over \AllRCodeRnd and
\AllNativeCodeRnd R and native code respectively. Unlike other open code
repositories such as GitHub, CRAN is a curated repository. Each submitted
package must abide to a number of well-formedness rules that are automatically
checked asserting certain quality. Most relevant for this work is that all of
the runnable code (including code snippets from examples and vignettes) is
tested and only a successfully running package is admitted in the archive.

We have downloaded and installed all available CRAN packages. Out of the
\AllCranRnd packages, we managed to install \AllLoadableRnd. The main reason for
this is that \rdt is based on R 3.5.0 and some of the packages are no longer
compatible with this version. Some packages also require extra native
dependencies which were not present on our servers. We defined two criteria for
including a package into the corpus:
\begin{inparaenum}[(1)]
  \item have a runnable code that covers a significant part of the package
  source code from which type signatures could be inferred, and
  \item have some reverse dependencies that will allows us to evaluate the
  inferred types using the runnable code from these dependencies.
\end{inparaenum}
The concrete thresholds used were: at least \ThresholdCodeCoverage of expression
coverage and at minimum \ThresholdRevdeps reverse dependencies. The code
coverage was computed for each package using
\covr\footnote{\url{https://github.com/r-lib/covr}}, the R code coverage R tool.
The reverse package dependencies were extracted from the package metadata using
builtin function.

The \CorpusLoadable selected packages contain \CorpusRunnableCodeRnd lines of
runnable code in examples (\CorpusExamplesCodeRnd), tests (\CorpusTestsCodeRnd)
and vignettes (\CorpusVignettesCodeRnd). Running this code results on average in
\CorpusMeanExprCoverage package code coverage (CRAN average is
\AllMeanExprCoverage). Together, there is \CorpusRevdesRnd (on average
\CorpusMeanRevdes, median \CorpusMedianRevdes; CRAN average is \AllMeanRevdes,
median \AllMedianRevdes) reverse dependencies with \CorpusRevdepRunnableCodeRnd
runnable lines of code resulting in \CorpusRevdepMeanCodeCoverage coverage (on
average) of the corpus packages.

Together there are \CorpusFunctionsRnd defined R functions
(\CorpusPublicFunctionsRnd are from packages' public API).
\CorpusSThreeFunctionsRnd are S3 functions. Packages in the corpus define
\CorpusSThreeClasses S3 classes.


The code coverage was computed for each package using
\covr\footnote{\url{https://github.com/r-lib/covr}}, the R code coverage R
tool.  The reverse package dependencies were extracted from the package
metadata using builtin function.

\paragraph{Type usage}%%%%%%%%%%%%%%%%%%%%%%%%%%%%%%%%%%

Table~\ref{ctb} reports on the 10 most frequent types
occuring in the corpus. Scalar doubles account for 11\% function arguments,
whereas logicals represent 27\% of observed values.  Together, \lgl and \chr
account for 60\%.


\begin{table}[!h]
% BEGIN Autogenerated
\small
\begin{tabular}{lrrrr}
\toprule
Type & Args & \% of Args & Observations & \% of Obs.\\
\midrule
\dbl & 13,255 &     11.0 & 41,560,797 & 1.5\\
\lgl & 10,142 &      8.4 & 712,895,959 & 27.2\\
\null & 9,175 &      7.6 & 122,202,183 & 4.6\\
\chr & 9,171 &       7.5 & 794,058,588 & 30.3\\
\VEC\dbl & 7,811 &   6.5 & 29,272,232 & 1.1\\
\VARGS & 7,759 &     6.4 & 76,369,333 & 2.9\\
\any & 6,243 &       5.2 & 255,539,497 & 9.0\\
\VEC\chr & 4,576 &   3.8 & 88,233,523 & 3.3\\
\CLASS{matrix} & 4,522     & 3.7 & 32,452,555 & 1.2\\
\CLASS{data.frame} & 2,781 & 2.3 & 7,749,094 & 0.3\\
\bottomrule
\end{tabular}

% END Autogenerated
\caption{Top types of arguments in R}\label{ctb}
\end{table}

%%%%%%%%%%%%%%%%%%%%%%%%%%%%%%%%%%%%%%%%%%%%%%%%%%%%%%%%%%%%%%%%%%%%%%%%%%%%%%
\section{Evaluation} %%%%%%%%%%%%%%%%%%%%%%%%%%%%%%%%%%%%%%%%%%%%%%%%%%%%%%%%%

\newcommand{\FIX}[1]{{\bf #1}\xspace}

\subsection{Expressiveness}

The first part of our evaluation attempts to shed light on how good a fit
our proposed type system is with respect to common programming patterns
occurring in widely used R libraries.

First we look at the share of monomorphic arguments and function signatures.
Monomorphic in this context means that the type is not relying on \any or
including a union.  The import of monorphism in this context is that it
means our type language can accurately capture an argument's type or a
function signature.

\begin{figure}[!h]  \centering
  \includegraphics[width=\linewidth]{plots/union_frequency.pdf}
  \caption{Size of unions}\label{fig:unionfreq}
\end{figure}

\ref{fig:unionfreq} shows the number of types inferred at each position of a
function signature. Most functions admit or return a uni-typed value. There are
\PercUnitypedPositions uni-typed parameter and return value positions in
function signatures. Only \PercManytypedPositions positions have more than three
types ascribed to them.

\begin{table}[H]
  \centering
  \resizebox{.5\linewidth}{!}{
    \begin{tabular}{lrrr}
      \toprule
      Types & Parameter \# & \% & Cumulative \%\\
      \midrule
      scalar & 28045 & 35.80 & 35.80\\
      class & 18510 & 23.63 & 59.43\\
      vector & 10296 & 13.14 & 72.58\\
      \rowcolor{lightgray}  \textcolor{black}{...} & \textcolor{black}{6611} & \textcolor{black}{8.44} & \textcolor{black}{81.02}\\
      \rowcolor{lightgray}  \textcolor{black}{any} & \textcolor{black}{5993} & \textcolor{black}{7.65} & \textcolor{black}{88.67}\\
      \addlinespace
      null & 4795 & 6.12 & 94.79\\
      \rowcolor{lightgray}  \textcolor{black}{list} & \textcolor{black}{3229} & \textcolor{black}{4.12} & \textcolor{black}{98.91}\\
      \textasciicircum{}vector & 461 & 0.59 & 99.50\\
      environment & 390 & 0.50 & 100.00\\
      \bottomrule
    \end{tabular}}
  \caption{Singleton Type Categories}\label{fig:singlefreq}
\end{table}

\ref{fig:singlefreq} provides a breakdown of the different type categories
observed at uni-typed positions. Scalar, class and vector are
the most common type categories. The shaded rows correspond to polymorphic
types. Removing those gives us \CountMonomorphicPositions
(\PercMonomorphicPositions) monomorphic positions in a corpus of \CountPositions
parameter and return positions.

\begin{figure}[!h]  \centering
  \includegraphics[width=\linewidth]{plots/function_polymorphism.pdf}
  \caption{Function Polymorphism}\label{fig:funpoly}
\end{figure}

\ref{fig:funpoly} shows the distribution of functions against the number of
polymorphic positions. We observe that \PercMonomorphicFunctions
(\CountMonomorphicFunctions) functions are monomorphic. The remaining
\PercPolymorphicFunctions (\CountPolymorphicFunctions) have at least one
polymorphic parameter or return type. Polymorphic functions are spread across
many packages, \CountMonomorphicPackages out of the 412 packages from our corpus
export only monomorphic functions.


Figure 1 shows the number of alternatives in the inferred type of function
parameter and return values.

XXX\% of the parameter positions are monomorphic.  \typetracer observes XXX
calls to XXX functions with XXX arguments. This translates to XXX parameters
and XXX return types.  A parameter or return type is from unioning all the
distinct observed types of

Figure 1 shows the distribution of parameter and return types for these
functions.

class is the most common type in our corpus, accounting for XXX\% of all the
inferred types. This is followed by an almost equal proportion,
28\%, scalar types and, surprisingly, only 13\% vector types. Vectors are the
workhorse of R, most primitive operations in R are vectorized and literals are
treated as single element vectors. It is unexpected to observe twice as many
scalars as vectors. It is even more unexpected to observe class as the most
common type. class of an object is used for dynamic dispatch by the S3 object
system in R. This suggests that usage of S3 is prevalent in our corpus.

null types. Together, these four types make up for 80\% of the observed
types in our corpus.  The other types, occurring only XXX\% of the times
include XXX environment types, XXXX expression types, XXXX pairlist types
and lastly, XXX externalptr types.

In order to count the occurrences of each type, we subdivided them into {\it
  kinds}: For our purposes, the {\it kind} of a type is equivalent to it's
top-level type constructor, so all classes have kind ``class'', all lists
have kind ``list'', and so on.  We found that classes are the most common
kind of type, accounting for roughly 31\% of types.  In order, the most
common classes are matrices (12\%), data.frames (7.5\%), formulas (2\%),
factors (2\%), and tibbles (2\%).  Roughly 25\% of classes are part of R's
base language implementation, the others being defined as part of some
package's functionality.

Besides classes, scalars and vectors are the next most common kind, making
up 41\% of types. with scalars making up 28.1\% of types and vectors
12.65\%.  Nulls and lists follow at 8\% and 7\% respectively, and the vararg
type \ldots makes up 6.5\% of arguments.  This all totals up to over 90\% of
types.  We will discuss the any type in more detail in
Section~\ref{subsec:any}.

The polymorphism in R is best described by {\it ad hoc} polymorphism,
wherein polymorphism is implemented by function and operator overloading
(see S3 dispatch) rather than in some more fundamental way in the language
design.  For the purposes of this evaluation, we will say that an argument
is polymorphic if its type is a union of two or more types, in other words
the argument has been observed to take more than one type of value.

We found that 15\% of argument positions were polymorphic, and that X\% of
functions were polymorphic.

Our type annotation language does not support sophisticated class typing,
and much of the polymorphism in R is due to these classes, namely through
the S3 dispatch mechanism.  In future work, we aim to more fully handle S3
and objects in R.  For now, if we count out


\subsubsection{Discussion}

Now, we aim to enhance the quantitative analysis of our language design by looking at select code snippets from the corpus we analyzed.
We will highlight some of the design decisions that we made, give code examples, and accompany them with function signatures generated by \typetracer.

\paragraph{NAs}
Our data supports the choice of making the presence of NAs explicit.  In
practice, programmers explicitly check for NAs in primitive vectors,
possibly sanitizing them if they are present. 

\begin{lstlisting}[basicstyle=\footnotesize\ttfamily]
#' @type <x: dbl[], n: dbl[], conf.level: dbl, maxsteps: dbl,
#'         del: dbl, bayes: lgl, plot: lgl, ...> => class<data.frame>
binom.profile <- function(x, n, conf.level = 0.95, maxsteps = 50,
                            del = zmax / 5, bayes = TRUE, plot = FALSE, ...) {
   xn <- cbind(x = x, n = n)
   ok <- !is.na(xn[, 1]) & !is.na(xn[, 2])
   x <- xn[ok, "x"]
   n <- xn[ok, "n"]
\end{lstlisting}

The \code{binom} package is a popular R package for computing confidence
intervals, and the \code{binom.profile} function does so using the profile
likelihood.  This code snippet highlights a data sanitization pattern: the
programmer first binds the vectors into a matrix, with one column for each
of \code{x} and \code{n}, then finds rows where both columns are not \NA,
then extracts non-\NA values and stores them into \code{x} and \code{n}
respectively.We see from the type that arguments \code{x} and \code{n} are intended to be free of \NA values.
\contractr will check them for compliance, and warn programmers of function misuse.

\paragraph{Scalars}
The data also suggests that programmers often use scalars, and do
dimensionality checks on their data.

\begin{lstlisting}[basicstyle=\footnotesize\ttfamily]
#' @type <n: dbl, x: int[]> => class<matrix>
hankel.matrix <- function( n, x ) { 
  ### n = a positive integer value for the order of the Hankel matrix
  ### x = an order 2 * n + 1 vector of numeric values
    ...
    if ( n != trunc( n ) )  stop( "argument n ix not an integer" )
    if ( !is.vector( x ) )  stop( "argument x is not a vector" )
    m <- length( x )
    if ( m < n )  stop( "length of argument x is less than n" )
\end{lstlisting}

The \code{hankel.matrix} function takes two arguments.  The function has a
number of interesting checks: \code{n} undergoes an integer type check;
\code{x} undergoes a vector type check; and \code{n} is expected to be a
scalar, as if it were a vector, the code \code{m < n} would essentially
check \code{m < n[i]} for each element in \code{n}, returning a vector of
comparison results, and the conditional of an if must be a scalar else R
generates a warning. The type for this function indicates that \code{x} must be
a vector, though here the type for \code{n} is found to be \dbl---this is likely due
to numeric values defaulting to \dbl type in R, as e.g. \code{typeof(2)} is \dbl.

\paragraph{Matrices}

Matrices represent another layer of complexity on top of vectorized
primitives.  Internally, matrices are simply vectors with class ``matrix''
and a \code{dims} attribute indicating matrix dimensions, and while not
codified in the language semantics, many internal functions will coerce
vectors to matrices automatically.  Relying on automatic coercion is unwise,
and many functions check to ensure that their arguments are matrices.

\begin{lstlisting}[basicstyle=\footnotesize\ttfamily]
#' @type <x: class<matrix>, w: ? ^dbl[], rows: null, cols: null, na.rm: lgl, ...> => ^dbl[]
rowWeightedMeans <- function(x, w = NULL, rows = NULL, cols = NULL,
                             na.rm = FALSE, ...) {
  if (!is.matrix(x)) 
    .Defunct(msg = sprintf("Argument 'x' should be a matrix.)
\end{lstlisting}

The \code{rowWeightedMeans} function calculates the weighted means of rows
or columns of a matrix \code{x}.  This function exhibits good practice in
type checking arguments, rightly producing a message to the user indicating
that an explicit matrix should be passed. We captured this with our class types, 
and indeed \typetracer inferred a good type for this function, allowing \contractr to 
perform the check present in this code automatically, notifying the users passing ill-typed arguments.

Thus far we've seen instances where we introduce types that subsume existing
argument validation code, but our intention is not to subsume {\it all} such
code. Consider:

\begin{lstlisting}[basicstyle=\footnotesize\ttfamily]
rowWeightedMeans <- function(x, w = NULL, rows = NULL, cols = NULL, na.rm = FALSE) {
    ...
    n <- ncol(x)
    if (length(w) != n) 
       stop("The length of 'w' does not match columns in 'x': ")
\end{lstlisting}

This code is from later in the same \code{rowWeightedMeans} function
discussed previously.  Here, the length of the weights vector \code{w} is
being checked for equality with the number of columns in the matrix
\code{x}.  Specifying this relationship as a type amounts to {\it dependent
  typing}, which we believe too complex to retrofit onto a dynamic data
science language such as R.  Ultimately, we decided to leave these kinds of
checks up to the programmer.

\paragraph{Structs}

In R, lists can have a names attribute, allowing lists elements to be accessed by name using the
\code{$} operator.

\begin{lstlisting}[basicstyle=\footnotesize\ttfamily]
#' @type <x: class<aov, lm> | class<lm>> => dbl;
agricolae::cv.model <- function(x) {
    suma2 <- sum(x$residual^2)
    gl <- x$df.residual
    promedio <- mean(x$fitted.values)
    return(sqrt(suma2/gl)*100/promedio)
}
data(sweetpotato)
model<-aov(yield~virus, data=sweetpotato)
cv.model(model)
\end{lstlisting}

In the code snippet, the function \code{cv.model} takes an argument
\code{x}, which we observed to always be a {\it linear model} (either ANOVA with \CLASS{aov, lm} or a simple linear regression model with \CLASS{lm}) which is part
of R's built-in statistics functionality.  Internally, linear models are
represented as lists with named elements, and one may access these named
elements using the \code{$} syntax, as seen on lines 2-4 in the snippet.  To
describe these kinds of values, we intended to include a struct type in our type
system, allowing users to specify name, type pairs as part of list type
definitions, though we found this to be problematic.

The inclusion of structs complicated our analysis by
polluting the data, illustrated on line 8 with the call
\code{data(sweetpotato)}.  It is quite common for R example code to use some
built-in data sets, such as \code{sweetpotato}, to show off some package
functionality with ``real data''.  When \typetracer is configured to detect
structs we saw a marked increase in misrepresentative types, as the types of
functions which use these data sets (such as \code{aov} in the code example) became polluted with the names from the example
sets.  We implemented a unification strategy for struct types, removing them
if they co-occurred with list types, and turning them into lists if two
co-occurring struct types shared no common names, but even so we found
ourselves with function signatures that were full of noise introduced by the
struct names.

We found that struct-like data was best described with classes, as illustrated in the code snippet, and many structs actually had a class.
Even though \code{x} is used like a struct, it in fact has a linear model class, which we believe is a more informative type.

%
\paragraph{Data Frames}

One of the most popular classes in R is the \code{data.frame} class, making up nearly 8\% of observed classes.  Data
frames and the derivative \code{tibble} (from the \code{tidyverse}
ecosystem) and \code{data.table} (from the \code{data.table} package) types
underpin a huge amount of data analysis in R.

One way to deal with data frames is through the struct type, with a named
field for each column of the data frame, but as mentioned previously structs
introduced undue noise into \typetracer's analysis results.  Further
complicating data frames is that many functions built to operate on them
operate in a name-agnostic way.  For instance, the \code{tidyverse} package
ecosystem allows programmers to pass column names to functions which operate
on their data frames.  In base R, idiomatic data frame use is to use string
column names to select rows from the frame (unless only a single column is
of interest, wherein the \code{$} syntax is appropriate).

\begin{lstlisting}[basicstyle=\footnotesize\ttfamily]
data(cars)
# tidyverse: Filter rows where speed is > 20. 
fast_cars_tidy <- filter(cars, speed > 20)
# base R: Get all rows where the "speed" column has a value > 20.
fast_cars_base <- cars[cars[, "speed"] > 20, ] # base example
\end{lstlisting}


Doing justice to this type will be an endeavour in and of itself, requiring
reconciliation of base R's \code{data.frame}, \code{tidyverse}'s
\code{tibble}, and \code{data.table}'s \code{data.table}.  Developing useful
types for the \code{tidyverse} ecosystem may well involve

It will also require proper treatment of function types, and perhaps even
parametric types, which dynamic analyses such as \typetracer are unable to
discern soundly.

In the context of our type annotation language, we ascribe the type \CLASS{data.frame} to \code{data.frame}-typed values.
Consider the following code snippet, showing the type signature for \code{dplyr}'s \code{all_equal} function which checks two \CLASS{data.frame} for equality.
\begin{lstlisting}[basicstyle=\footnotesize\ttfamily]
#' @type <target: class<data.frame>, current: class<data.frame>, ignore_col_order: lgl, 
#'         ignore_row_order: lgl, convert: lgl, ...> => (chr | lgl)
dplyr::all_equal <- function(target, current, ignore_col_order = TRUE,
                      ignore_row_order = TRUE, convert = FALSE, ...) { ... }
\end{lstlisting}

\paragraph{Objects}

R has multiple, disparate object systems, with the language providing direct
support for S3, S4, and R5.  Thus far, we've mainly discussed S3 and its
dispatch mechanism, and indeed our class types capture the S3
classes of values, but not user-defined polymorphism.  We will not discuss R5 as it is still under development.

S4, like S3, uses the class attribute to store types, we type them with a \CLASS{...} type.  
We stuck to simple class-based types for S4 since
doing anything more would be immensely complicated, and outside of the scope
of this work.  For instance, the mechanics of S4 dispatch are more complex
than for S3 (S4 dispatches on the class of {\it all} arguments), and users
can define their own class hierarchies that we would need to incorporate in
our type analysis and contract checking frameworks.  Further,we found
  limited use of S4 during our analysis. Coming up with a type system that
accounts for all of these factors and consolidates multiple
object-orientation frameworks in a single language design is an interesting
problem in and of itself, and we aim to extend our type annotation framework
to account for this in future work.  For now, simple class types for S3 and S4 suffice.

To illustrate informative class types, consider the following code snippet.
\code{distr} is a package which implements statistical distributions in an object-oriented way, and the \code{Binom} function acts as a constructor for \CLASS{Binom} values.
Another example is from the \code{Epi} package, which provides functionality for demographic and epidemiological analysis in the Lexis diagram.
The function we highlighted, \code{timeBand}, extracts time band data from the passed Lexis object.
We see from the type that the first argument is the Lexis object, and that either a \VEC\dbl or a \CLASS{factor} is returned.
Often, class members are implemented as attributes on R values, and it's likely that this is the case for Lexis objects as evidenced by the programmer accessing the "breaks" attribute in the function code.

\begin{lstlisting}[basicstyle=\footnotesize\ttfamily]
#' @type <size: dbl, prob: dbl> => class<Binom>
distr::Binom <- function(size = 1,prob = 0.5) new("Binom", size = size, prob = prob)

#' @type <lex: class<`Lexis`, `data.frame`>, time.scale: chr, type: chr> => 
#'           class<`factor`> | dbl[];
Epi::timeBand <- function(lex, time.scale, type="integer") {
  time.scale <- check.time.scale(lex, time.scale)[1]
  breaks <- attr(lex, "breaks")[[time.scale]]
  ...
\end{lstlisting}


\subsection{Robustness}

In the last section, we evaluated our type annotation language based on a
set of function types that were automatically generated by \typetracer.
Now, we're interested in knowing how representative our trace-generated
types are of the functions they were generated for.  To measure this, we
conducted a large-scale experiment: for each package in the corpus discussed
in Section~\ref{sec:corpus}, we ran \typetracer on its test, example, and
vignette code, obtaining types for package functions. Then, we
ran the test, example, and vignette code of each of the package's reverse
dependencies (or clients).  We collected failed contract assertions in order
to analyze them and perhaps identify weaknesses in our type annotation
language.

We ran our evaluation on \NUMPKGSEVAL packages and recorded \NUMASSERTIONS
total assertions.  Overall, we found that only \PERCFAILEDASSERTIONSWUNDEF
of contract assertions failed.  We made a simplification in our analysis,
capping the number of function arguments we generate types for to 20, and
this resulted in only \PERCASSERTIONSUNDEF failed assertions, which suggests
that the simplification we made did not greatly affect the results.  Once we
controlled for these, we found that \PERCFAILEDASSERTIONS of all assertions
failed for reasons related to our generated types.  Additionally, we found
that \PERCSUCCARG of parameter types and \PERCSUCCFUNS of functions types
were never violated.  The number of immaculate function types increases to
\PERCSUCCFUNNOSTHREE if we discount S3 functions: These S3 functions
represent user-defined polymorphism which we do not tackle in this work.
Overall, these numbers are promising, and suggest that the type signatures
generated by running \typetracer on our corpus are indeed representative,
and capture intended function behaviour.


We breakdown the failed assertions by type in
Table~\ref{tbl:fail-breakdown-0}.  
Accounting for 27.37\% of assertion failures, the most common vilation arises when a value with \CLASS{matrix} is passed where a \VEC\dbl is expected.
Considering these types, we might imagine them to be compatible, however not allowing this coercion was a deliberate design decision:
coercion from vectors to matrices is ad hoc at best, and surely not a
practice codified in the language.  In a similar vein, another popular
failing assertion is checking if a ${\bf dbl}[]$ has type ${\bf int}[]$,
another case of commonly performed coercion.  We did not include these types
of coercions in our type annotation framework as programmers cannot rely on
them, and it is not always the case that the coercions are safe to perform.

\begin{table}
% BEGIN Autogenerated

\begin{tabular}{llrrr}
\toprule
Passed & Arg Type & Occurrences & \% Total & Cumulative \%\\
\midrule
\VEC\dbl & \CLASS{matrix} & 371,896 & 27.37 & 27.37\\
\VEC\dbl & \VEC\int & 117,212 & 8.62 & 35.99\\
\chr & \CLASS{bignum} | \VEC\raw & 100,100 & 7.37 & 43.36\\
\dbl & \int | \null & 60,578 & 4.46 & 47.81\\
\VEC\dbl & \CLASS{timeSeries} & 58,872 & 4.33 & 52.15\\
\addlinespace
\CLASS{matrix} & \CLASS{timeSeries} & 55,005 & 4.05 & 56.19\\
\VEC\dbl & \dbl & 53,338 & 3.92 & 60.12\\
\dbl & \CLASS{data.frame} & 32,553 & 2.40 & 62.51\\
\VEC\dbl & \CLASS{data.frame} & 31,576 & 2.32 & 64.84\\
\dbl & \int & 20,586 & 1.51 & 66.35\\
\bottomrule
\end{tabular}
% END Autogenerated
\label{tbl:fail-breakdown-0}
\caption{Top types of contract assertion failures.}
\end{table}


In addition to the raw number of failing contract checks, we were interested
in how many functions had a parameter where a contract check failed, and
overall we found this to be the case in \PROPFUNSFAILEDCHECK of functions.
Digging a little deeper, we noticed that many of these functions were
performing S3 dispatch, which is part of R's object-orientation framework:
thus, these functions implement user-defined polymorphism which we do not
handle in our type framework.  Removing those functions, we see that
\PROPFUNSFAILEDCHECKNOSTHREE of functions exhibited failing contract checks.
These are functions which were under-tested, as calls to these functions
represent only \PERCCALLSBADFUNSINTESTS of recorded calls during
\typetracer's run on the core corpus to infer types.

Turning our attention now to arguments, we found that only
\PROPARGSFAILINGASSERTS of function parameters exhibited a failed assertion.
Table~\ref{tbl:fail-breakdown-0} showed the raw results from counting
runtime occurrences, but that data alone does not tell the full story, as
some failures may be overrepresented if e.g. a failing contract assertion
was in a loop.  We were interested in knowing for each of the most common
violations in Table~\ref{tbl:fail-breakdown-0}, how many different arguments
had that type, and how many of those exhibited the contract failure in
question.  Table~\ref{tbl:fail-breakdown-1} breaks down the failed assertions by type,
folding away multiple identical failed contract assertions for the same parameter
position.  The first row of this table reads: a value of type
\VEC\dbl was passed to 15 different function parameters expecting a
\CLASS{matrix}, of which there are 1394 in total: 1.08\% of \CLASS{matrix}-typed parameters
were passed \VEC\dbl values.

We see that even though the double vector and matrix issue was wildly
prevalent in the raw, dynamic contract evaluation numbers, the number of
actual function argument types that were violated is very small.  The story
is similar with the double and integer coercion we mentioned earlier: it
represents many dynamic contract failures, but very few of the \VEC\int-typed 
arguments have their contracts violated by \VEC\dbl-typed
values.  Rows 5 and 6 are interesting: we see that rather often arguments
expecting \CLASS{timeSeries} data are passed \VEC\dbl or \CLASS{matrix} values.  This
is a quirk of the \code{timeSeries} package, whose functions often accept
matrices and vectors, converting them to time series in an ad hoc manner.
Note that the code coverage of the \code{timeSeries} tests, examples, and
vignettes on the \code{timeSeries} package code is only 58.24\%, which is
one possible explanation of why these contract failures are occurring: the
types that \typetracer generates are only as good as the test code its run
on.

 \begin{table}
% BEGIN Autogenerated

\begin{tabular}{llrrr}
\toprule
Passed & Arg Type & \# Args Failed & \# Args with Type & \% Failure\\
\midrule
\VEC\dbl & \CLASS{matrix} & 15 & 1394 & 1.08\\
\VEC\dbl & \VEC\int & 44 & 657 & 6.70\\
\chr & \CLASS{bignum} | \VEC\raw & 1 & 3 & 33.33\\
\dbl & \int | \null & 5 & 20 & 25.00\\
\VEC\dbl & \CLASS{timeSeries} &19 & 35 & 54.29\\
\addlinespace
\CLASS{matrix} & \CLASS{timeSeries} &  19 & 35 & 54.29\\
\VEC\dbl & \dbl & 100 & 5021 & 1.99\\
\dbl & \CLASS{data.frame} &  1 & 900 & 0.11\\
\VEC\dbl & \CLASS{data.frame} &  6 & 900 & 0.67\\
\dbl & \int & 48 & 529 & 9.07\\
\bottomrule
\end{tabular}

% END Autogenerated
 \label{tbl:fail-breakdown-1}
 \caption{Results from Table~\ref{tbl:fail-breakdown-0}, broken down by occurrences of the expected type as a parameter type.}
 \end{table}

Table~\ref{tbl:fail-breakdown-2} presents data on the most frequently
violated contracts amongst the most frequently occurring argument types.  We
selected argument types which were in the 90th percentile of argument type
occurrences, computed the most frequent type signature violations among
them, and reported the most frequently violated contracts together with the
type of the value that violated that contract.  The first row of the table
reads: 44 function arguments with \VEC\int type are passed \VEC\dbl  instead, and 657 arguments have \VEC\int type.  Had we failed
to capture some key usage pattern of R with our type annotation framework,
we would likely see it here, and we can see this in action if we consider
Table~\ref{tbl:fail-breakdown-3}, which was obtained identically to
Table~\ref{tbl:fail-breakdown-2} with a percentile of 80th percentile
instead.  The most frequent argument type violation pattern in that of \CLASS{data.frame} values
 passed to arguments expecting tibbles, which are an extension of \CLASS{data.frame} built in the \code{tidyverse} package suite: 
 this occurs in only 22\%
of such arguments, and represents cases where tests did not adequately cover
all valid function inputs.  Separate from the issue of testing, we can
capture this behaviour with user-defined subtyping, as data frames are
subtypes of tibbles, but this necessitates a proper object-oriented type
system for R which is outside the scope of this work.

 \begin{table}
% BEGIN Autogenerated

\begin{tabular}{llrrr}
\toprule
Passed & Arg Type & \# Args Failed & \# Args with Type & \% Failure\\
\midrule
\VEC\dbl & \VEC\int & 44 & 657 & 6.70\\
\NAVEC\lgl & \null & 158 & 2756 & 5.73\\
\VEC\chr & \chr & 183 & 4102 & 4.46\\
\chr & \null & 117 & 2756 & 4.25\\
\dbl & \null & 107 & 2756 & 3.88\\
\bottomrule
\end{tabular}
% END Autogenerated
 \label{tbl:fail-breakdown-2}
 \caption{What is the highest failure rate among popular argument types? For argument signatures whose frequency is in the 90th percentile.}
 \end{table}
 
  \begin{table}
% BEGIN Autogenerated

\begin{tabular}{llrrr}
\toprule
Passed & Arg Type & \# Args Failed & \# Args with Type & \% Failure\\
\midrule
\CLASS{data.frame} & \CLASS{data.frame, tbl, tbl\_df} & 63 & 284 & 22.18\\
\CLASS{matrix} & \CLASS{array} & 13 & 70 & 18.57\\
\NAVEC\lgl & \lgl | \null & 9 & 49 & 18.37\\
\VEC\chr & \LIST{\VEC\chr} & 20 & 109 & 18.35\\
\VEC\clx & \clx & 7 & 42 & 16.67\\
\bottomrule
\end{tabular}
% END Autogenerated
 \label{tbl:fail-breakdown-3}
 \caption{What is the highest failure rate among popular argument types? For argument signatures whose frequency is in the 80th percentile.}
 \end{table}
 
In sum, we believe that this evaluation shows that the type signatures we
generate from traces are quite good.  Only \PERCFAILEDASSERTIONSWUNDEF of
contract assertions failed at runtime, representing failures in as few as
\PROPARGSFAILINGASSERTS of argument types.  Even though
\PROPFUNSFAILEDCHECKNOSTHREE of functions had at least one argument type
involved in a failing contract check, these functions are under-tested,
representing only \PERCCALLSBADFUNSINTESTS of calls observed while inferring
types.
 
%
%
%
%
\subsection{Usefulness of the Type Checking Framework}

There is a number of ways in R to check types of function parameters. The
default and the most common way\footnote{\CranStopifnotRatio of all runtime
  checks in the whole of CRAN} is to use \code{stopifnot} function from the R
base package. It takes a number of R expressions which all should evaluate to
true otherwise a runtime exception is thrown with a message quoting the
corresponding failed expression. For example, the following code checks whether
a given parameter \code{x} is a scalar string:
%
\begin{lstlisting}
stopifnot(is.character(x), length(x) == 1L, !is.na(x))
\end{lstlisting}

Next to \code{stopifnot}, there are 4 packages in CRAN\footnote{Packages are
  available on CRAN website: \url{https://cran.r-project.org/web/packages/}}.
that focus on runtime assertions: \emph{assertive}, \emph{ensurer},
\emph{assertr} and \emph{assertthat} \emph{assertive} and \emph{ensurer} have
not been updated 2016 and 2015 respectively. \emph{assertr} is used by only
\AssertrRevdeps other packages and in the current version focuses on checking
properties of data frames. Only the \emph{assertthat} is maintained and used
(with \AssertthatRevdeps reverse dependencies). The advantage of
\emph{assertthat} over the R's default is that it provides a much better error
message.

One way to asses the usefulness of our type checking system is to find out how
many of the existing type checking constrains could be replaced by \contractr.
To measure this, we have extracted all calls to \code{stopifnot} and
\code{assertthat} assertions and checked which one of them could be either
completely replaced by \contractr or at least partially simplified by removing a
portion of an assertion expression. This is useful, because a common pattern is
that the first part of the parameter assertion checks its type while the rest
checks its value. In the example above, the whole expression could be replaced
by \code{character} type.

Out of the \CorpusLoadable packages, \CorpusAssertsInPackages use runtime
assertions. Together there are \CorpusAsserts asserts in
\CorpusAssertsInFunctions functions. Among these, \contractr can replace
\CorpusTypedAsserts (\CorpusTypedAssertsRatio) assertion calls across
\CorpusTypedAssertsPackages packages and \CorpusTypedAssertsFunctions functions.
Furthermore, additional \CorpusPartiallyTypedAsserts
(\CorpusPartiallyTypedAssertsRatio) asserts in packages
\CorpusPartiallyTypedAssertsPackages packages and
\CorpusPartiallyTypedAssertsFunctions functions could have been simplified.

Checking the type of function parameters is not something that is seen often in
the R code. In the whole of CRAN, there are only \CranAssertsRnd asserts in
\CranAssertsInFunctionsRnd functions define in \CranAssertsInPackagesRnd
packages. One might speculate that one of the reasons is that the current way is
not convenient as it is rather verbose. Our system can infer type annotations
for existing functions automatically. This can remove or partially remove over
\CorpusPartiallyTypedAssertsRatio of existing assertions.

\section{Future Work}

We see two major avenues for future work.

\subsection{Types for R Objects}

\subsection{Types for Data Frames}

\section{Conclusion}



\ldots


\bibliography{bib/biblio,bib/jv,bib/r,bib/new,bib/gradual}
\end{document}
